\documentclass[a4paper,11.5pt]{article}
\usepackage[textwidth=170mm, textheight=230mm, inner=20mm, top=20mm, bottom=30mm]{geometry}
\usepackage[normalem]{ulem}
\usepackage[utf8]{inputenc}
\usepackage[T1]{fontenc}
\PassOptionsToPackage{defaults=hu-min}{magyar.ldf}
\usepackage[magyar]{babel}
\usepackage{amsmath, xcolor, amsthm,amssymb,paralist,array, ellipsis, graphicx}
%\usepackage{marvosym}

\usepackage{listings}
\lstset{
	language=C++, 
	basicstyle=\ttfamily, 
	keywordstyle=\color{blue}\ttfamily, 
	stringstyle=\color{red}\ttfamily,
	tabsize = 4
}

\makeatletter
\renewcommand*{\mathellipsis}{%
	\mathinner{%
		\kern\ellipsisbeforegap%
		{\ldotp}\kern\ellipsisgap%
		{\ldotp}\kern\ellipsisgap%
		{\ldotp}\kern\ellipsisaftergap%
	}%
}
\renewcommand*{\dotsb@}{%
	\mathinner{%
		\kern\ellipsisbeforegap%
		{\cdotp}\kern\ellipsisgap%
		{\cdotp}\kern\ellipsisgap%
		{\cdotp}\kern\ellipsisaftergap%
	}%
}
\renewcommand*{\@cdots}{%
	\mathinner{%
		\kern\ellipsisbeforegap%
		{\cdotp}\kern\ellipsisgap%
		{\cdotp}\kern\ellipsisgap%
		{\cdotp}\kern\ellipsisaftergap%
	}%
}
\renewcommand*{\ellipsis@default}{%
	\ellipsis@before
	\kern\ellipsisbeforegap
	.\kern\ellipsisgap
	.\kern\ellipsisgap
	.\kern\ellipsisgap
	\ellipsis@after\relax}
\renewcommand*{\ellipsis@centered}{%
	\ellipsis@before
	\kern\ellipsisbeforegap
	.\kern\ellipsisgap
	.\kern\ellipsisgap
	.\kern\ellipsisaftergap
	\ellipsis@after\relax}
\AtBeginDocument{%
	\DeclareRobustCommand*{\dots}{%
		\ifmmode\@xp\mdots@\else\@xp\textellipsis\fi}}
\def\ellipsisgap{.1em}
\def\ellipsisbeforegap{.05em}
\def\ellipsisaftergap{.05em}
\makeatother

\usepackage{hyperref}
\hypersetup{
	colorlinks = true	
}

\begin{document}
	%%%%%%%%%%%RÖVIDÍTÉSEK%%%%%%%%%%
	\setlength\parindent{0pt}
	\def\s{\hspace{0.2mm}\vphantom{\beta}}
	\def\Z{\mathbb{Z}}
	\def\Q{\mathbb{Q}}
	\def\R{\mathbb{R}}
	\def\C{\mathbb{C}}
	\def\N{\mathbb{N}}
	\def\Ra{\overline{\mathbb{R}}}
	
	\def\sume{\displaystyle\sum_{n=1}^{+\infty}}
	\def\sumn{\displaystyle\sum_{n=0}^{+\infty}}
	
	\def\narrow{\underset{n\rightarrow+\infty}{\longrightarrow}}
	\def\limn{\displaystyle\lim_{n\to +\infty}}
	\def\limx{\displaystyle\lim_{x\to +\infty}}
	
	\theoremstyle{definition}
	\newtheorem{theorem}{Tétel}[subsection] 
	
	\theoremstyle{definition}
	\newtheorem{definition}[theorem]{Definíció} 
	\newtheorem{example}[theorem]{Példa} 
	\newtheorem{task}[theorem]{Feladat} 
	\newtheorem{note}[theorem]{Megjegyzés}
	%%%%%%%%%%%%%%%%%%%%%%%%%%%%%%%%%%%%%%%%%%%%%%%%%%%%%%%%%%%%%%%%%%%%%
	\begin{center}
		{\LARGE\textbf{C++}}
		
		{\Large Gyakorlat jegyzet}
		
		10. óra.
	\end{center}
	A jegyzetet \textsc{Umann} Kristóf készítette \textsc{Horváth} Gábor  előadásán. (\today)
	\section{$+/-$ megoldás}
	Paraméter dedukció az, amikor a fordító ki tudja találni a template paraméterét egy függvénynek (nagyon dióhéjban). Példa:
	\begin{lstlisting}
template <class T>
void f(const T& param) {}

int main()
{
	f<std::string>("Hello"); //nincs dedukcio
	f("Hello"); //van dedukcio
}
	\end{lstlisting}
	
	A második feladat egy olyan template osztály létrehozása, amely heapen tárol egy objektumot, és reguláris típus. Mivel ez egy template osztály, nem tudjuk milyen típust kapunk, alapértelmezetten vegyük úgy, hogy a másolás költséges, így másolás helyett vegyünk át konstans referenciákat.
	\begin{lstlisting}
template <class T>
class X
{
	X(const T& val) : data(new T(val)) {}
	~X() {delete data;}
	X(const X &other) : data(new T(*other.data)) {}
	X &operator=(const X &other)
	{
		if (this == &other)
			return *this;
		delete data;
		data = new T(*other.data);
		return *this;
	}
private:
	T *data;
};

int main()
{
	X<int> a;
	X<int> y(a);
	y = a;
}
	\end{lstlisting}
	Bővebb magyarázathoz az 5-6. gyakorlat anyagát érdemes megnézni.
	
	\medskip
	Vizsgán gyakran (DE NEM MINDIG!) nem kell megírni az értékadó operátort meg copy konstruktort, stb, mert gyakran olyan adattagokat kell tárolnunk, mint az \texttt{std::vector}. Mivel az alapértlemezett copy konstruktor és értékadó operátor az adattagok copy konstruktorát és értékadó operátorát hívja meg, így reguláris típusokat tárolunk, akkor a mi típusunk is reguláris lesz.
	\section{STL (folytatás)}
	Három konténertípust különböztetünk meg az STL-ben:
	\begin{itemize}
		\item Szekvenciális: Olyan sorrendben tárolja az adattagokat, ahogyan betesszük őket. \\Példa: \texttt{std::vector}, \texttt{std::deque,} \texttt{std::list.}
		\item Asszociatív: Az elemek rendezettek a konténerben.\\ Példa: \texttt{std::set}, \texttt{std::map}, \texttt{std::multiset}, \texttt{std::multimap}.
		\item Konténer adapter: Egy meglévő konténert alakítanak át, általában szogorítanak. Példaképp ha vesszük az \texttt{std::deque} konténert, ami egy kétvégű sor, könnyen tudunk belőle egy egyvégű sort csinálni. Ezen a konténerek általában egy másik konténert tárolnak, és annak a funkcióit szigorítják.
		\\Példa: \texttt{std::queue}, \texttt{std::stack}.
	\end{itemize}
	\subsection{Funktorok}
	Emlékezzünk vissza a rendezésre. Ugye az \texttt{std::set}-ben van egy alapértelemezett, \texttt{std::string} esetben lexikografikus rendezés, de mi ezt felülírhatjuk egy funktorral. A funktor egy olyan osztály, melynek túl van terhelve a gömbölyű zárójel operátora. (tehát kvázi meg lehet hívni)
	
	\smallskip
	Egy egyszerű példa:
	\begin{lstlisting}
struct S
{
	int x;
	int operator()(int y)
	{
		return x + y;
	}
};
int main()
{
	S s1, s2;
	s1.x = 5;
	s2.x = 8;
	std::cout << s1(2) << std::endl; //7
	std::cout << s2(2) << std::endl; //10
}
	\end{lstlisting}
	Itt gyakorlatilag az objektum függvényként funkcionál. Ha nagyon furának tűnhet ez, jusson eszünkbe, hogy nagyon hasonló ez a \texttt{[]} operátorhoz. A funktorok azért fontosak, mert megadhatók template paraméterként, és használhatóak például rendezésre is. Példaképp vegyük az előző órán beszélt \texttt{std::set}-et: ennek két template paramétere van, az egyik a tárolandó típus, a másik pedig a rendezés (van még egyéb is, de ezzel most ne foglalkozzunk). Képzeljük el mondjuk így:
	
\begin{lstlisting}
namespace std
{
	template <class Value, class Compr = std::less<Value> >
	class set
	{
		//...
		
		//ossze kell hasonlitani ket elemet: v1 es v2.
		Compr c;
		if ( c(v1, v2) )
		
		//...
	};
}
\end{lstlisting}
	Figyeljük meg, hogy az \texttt{std::set}-nek van alapértelmezett rendezése is, ez az \texttt{std::less} (ld.: cppreference.com). A \texttt{set} egy olyan funktort fog várni, melyek a gömbölyű zárójel operátora két \texttt{Value} típusú objektumot vár paraméterként, és a visszatérési értéke \texttt{bool.}
	
	\smallskip
	Miért használunk funktorokat mondjuk függvénypointerek helyett? Világos, hogyha egyedi rendezést akarunk, akkor muszáj ezt az információt valahogy átadni. A funktorok erre alkalmasabbak, példaképp vegyük ehhez az \texttt{std::find\_if}-et. Enek is meg kell adni egy funktort: Hogyan tudnánk mi függvénypointerrel a második páros számot vele? Vagy az 5. 8-al osztható számot? Nos, függvényekkel igencsak nehezen (kb statikus adattagokra kényszerülnénk, vagy egy hasonlóan nem épp elegáns megoldásra), azonban egy funktorban létrehozhatunk egy számlálót, melyet tudunk szépen növelgetni.
	\smallskip
	
	Segíthet a megértésben, ha tekintjük az alábbi példát:
	\begin{lstlisting}
template <class Compr>
bool f(int a, int b, Compr c)
{
	return c(a, b);
}

struct Less
{
	bool operator()(int a, int b) const
	{
		return a < b;
	}
};

int main()
{
	if( f(2,3, Less()) )
		std::cout << "2 kisebb mint 3! ";
		
	int a = 2, b = 2;
	if ( f(a, b, Less()) == false && f(b, a, Less()) == false )
		std::cout << "a es b egyenloek!" << std::endl;
}
	\end{lstlisting}
	Kimenet: \texttt{2 kisebb mint 3! a es b egyenloek!}
	
	Ez a funktor (\texttt{Less}) egy un. \textit{binary predicate}, azaz a gömbölyű zárójel operátora két azonos típusú objektumhoz rendel egy logikai értéket. Figyeljük meg, hogyan adtuk meg a harmadik paramétert: \texttt{Less}-nek meghívtuk a default konstruktorát, de nem adtunk neki változónevet. Itt egy un. névtelen temporális változó jön létre, mely \texttt{f} paramétereinek kiértékelésekor jön létre, majd az \texttt{f} lefutása után meg is semmisül. Figyeljük meg azt is, hogy ebben az esetben érték szerint vettünk át: ez nem csak hatékonyabb, hisz \texttt{Less} mérete minimális (nincs adattagja), de mivel \texttt{Less()} egy jobbérték, csak így, vagy konstans referenciával tudnánk átvenni.

	\smallskip
	Jusson eszünkbe, mikor ekvivalens két elem! $a = b \quad \Leftrightarrow \quad \neg(a<b) \wedge \neg(b<a)$.
	\begin{lstlisting}
#include <iostream>
#include <set>

struct strlen
{
	bool operator()(const std::string &lhs, const std::string &rhs) const
	{
		return lhs.length() < rhs.length();
	}
};

int main()
{
	std::set<std::string, strlen> s;
	s.insert("C++");
	s.insert("Java");
	s.insert("Haskell");
	std::cout << s.count("ADA") << std::endl;
}
	\end{lstlisting}
	A \texttt{count} tagfüggvény megszámolja, hány \texttt{ADA}-val \textbf{ekvivalens} elem van a \texttt{set}ben. Mivel a funktorunk úgy hasonlít össze két elemet, hogy megnézi, melyik string hossza nagyobb,  \texttt{C++} és az \texttt{ADA} ekvivalens lesz, így legnagyobb meglepődésünkre 1-et fog ez a program kiírni!
	%miért pocakos a dékán úr? mert has kell
\begin{lstlisting}

int main()
{
	std::set<std::string, strlen> s;
	s.insert("C++");
	s.insert("Java");
	s.insert("Haskell");
	s.insert("ADA");
	std::cout << s.size() << std::endl;
}
\end{lstlisting}
	Most értelemszerűen azt várnánk, hogy a méret 4 legyen, de mégis 3at ír ki, mert az \texttt{insert} függvény nem teszi be az \texttt{ADA}-t, mert az a \texttt{C++}al ekvivalens.
\begin{lstlisting}
#include <iostream>
#include <set>

struct strlenWrong
{
	bool operator()(const std::string &lhs, const std::string &rhs) const
	{
		return lhs.length() <= rhs.length(); //ekvivalensek is lehetnek
	}
};

int main()
{
	std::set<std::string, strlenWrong> s;
	s.insert("C++");
	s.insert("Java");
	s.insert("Haskell");
	s.insert("ADA");
	std::cout << s.count("ADA") << std::endl;
	std::cout << s.size() << std::endl;
}
\end{lstlisting}
	Most figyeljük meg (emlékezzünk vissza dimatra), hogy egy szigorú rendezés helyett egy gyenge rendezést definiáltunk! Miért lehet ez problémás?
	
	\smallskip
	Helyettesítsünk be a rendezésünkbe.
	\[ \neg(a.length() \leq b.length())\quad \wedge \quad \neg(b.length() \leq a.length()) \]
	Itt ha Ha a \texttt{C++}t és az \texttt{ADA}t behelyettesítsük, látjuk, hogy ez a formula hamisat ad, így nem bizonyulnak majd ekvivalensnek.
	
	Ha nem szigorú részben rendezést használunk, hanem gyengét, akkor a reflexivitást is elvesztjük. Ez azt jelenti, hogy egy elem nem lehet ekvivalens önmagával!
	
	Ha lefordítjuk és futtatjuk ezt a programot, 0-t és 4-et fogunk kapni. Picikét hülye a rendezésünk, mivel ha ekvivalenciát vizsgálunk, sose fog igazat adni, sose tudjuk meg, egy adott benne van-e, az elemek törlése se teljesen jó így.
	\subsection{map}
	Előző órán volt erről szó, ez egy asszociatív kulcs-érték párokat tartalmazó konténer. Vannak azonban kellemetlen meglepetések, amelyek bajba juttathatnak:
\begin{lstlisting}
#include <iostream>
#include <map>

int main()
{
	std::map<std::string, int> m;
	std::cout << m.size() << std::endl;
	if(m["c++"] != 0)
	{
		std::cout << "nem 0" << std::endl;
	}
	std::cout << m.size() << std::endl;
}
\end{lstlisting}
	Itt a kapott kimenet \texttt{0 1} lesz. Ha visszamelékszünk, a \texttt{[]} operátor úgy működik, hogy ha a \texttt{map} nem tartalmazza a paraméterként kapott kulcsot, akkor létrehozza és beszúrja, ha benne van, visszaadja az értékét. Bár mi nem tettük bele azt hogy \texttt{c++} szándékosan, de pusztán azzal, hogy rákérdeztünk, akaratlanul is megtettük. (Figyeljük meg, hogy akkor tudnánk új elemet hozzátenni, ha azt írnánk pl. hogy \texttt{m["C++"] = 3}; azonban mi nem adtunk meg ehhez a kulcshoz értéket. Ilyenkor a map alapértelmezett értéket rendel a kulcshoz, szám típusoknál 0-t, így be fog szúrni egy \texttt{(c++, 0)} párt.)
	\begin{lstlisting}
bool f(const std::map<std::string, int> &m)
{
	//m["c++"] forditasi hiba: [] operator nem konstans fuggveny.
	
	return m.find("C++") != m.end();
}
	\end{lstlisting}
	Ez a függvény helyesen vizsgálja, hogy a \texttt{c++} benne van-e. (Ha cppreference-en rákeresünk, a \texttt{map} \texttt{find} tagfüggvénye egy iterátort ad vissza a talált elemre, illetve visszaadja a past-the-end iterátort ha nem találja meg.)
	\subsection{STL felépítése}
	Látogassunk el a \url{http://en.cppreference.com/w/} oldalra, és picikét barátkozzunk vele. Itt láthatjuk az összes STL konténert, de sok fontos információhoz is hozzájuthatunk, pl. az \texttt{std::string}-el kapcsolatban is: példaképp az \texttt{std::string} csak egy typename!
	
	Mivel teljesen irreális minden STL algoritmusról részletesen beszélni, így ajánlatos ezeket is nézni: \url{http://en.cppreference.com/w/cpp/algorithm}.
	
	Érdemes megbarátkozni az STL-el: lévén vizsgán lehet használni mindent ami a standard könyvtárban benne van, könnyen megspórolhatjuk a gondolkodást a megfelelő STL algoritmus hívásával.
	
	\medskip
	Rendezzük egy \texttt{vector} elemeit!
	\begin{lstlisting}
std::vector<int> v {6,3,7,4,1,3};
std::set<int> s {6,3,7,4,1,3};
std::set<int> s1(v.begin(), v.end());
v.assign(s1.begin(), s1.end());
for(int i : v)
{
	std::cout << i << ' '; // 1 3 4 6 7 
}
	\end{lstlisting}
	\begin{note}
		Az fenti inicializálások a c++11-es újítás részei, sok magyarázatra gondolom nem szorulnak.
		
		A fenti ciklus szokatlan lehet, ez is egy c++11es újítás, un. \textit{range-based for loop}. A fenti kód ezzel ekvivalens:
		\begin{lstlisting}
int i;
for(std::vector<int>::iterator it = v.begin(); it != v.end(); it++)
{
	i = *it;
	std::cout << i << ' ';
}
		\end{lstlisting}
		Számunkra legyen elég most annyi, hogy az így írt ciklusok akkor működnek, ha a jobb oldalon álló objektum (most \texttt{v}) rendelkezik \texttt{begin()} és \texttt{end()} tagfüggvényekkel, vagy pedig az objektum egy tömb. Ez nem a teljes lista, de most legyen egyenlőre ennyi elég.
	\end{note}
	Most kihasználtuk, hogy az \texttt{std::set} alapértelmezetten rendezett: átpakoljuk abba, majd vissza. No persze, ez minden, csak nem hatékony. Továbbá feltűnhet, hogy az egyik 3-as kiesett: az \texttt{std::set} kiszűrte az egyiket, mert ekvivalens elemet nem tárol. Ennél sokkal hatékonyabb, ha egy STL algoritmussal dolgozunk:
	\begin{lstlisting}
std::vector<int> v {6,3,7,4,1,3};
std::sort(v.begin(), v.end());
for(int i : v)
{
	std::cout << i << ' '; // 1 3 3 4 6 7 
}
	\end{lstlisting}
	Fontosabb algoritmusok:
	\begin{compactitem}
		\item find
		\item search
		\item sort
		\item reverse
		\item rotate
		\item unique
		\item remove
		\item stb...
	\end{compactitem}
	Beszéljünk is ez utóbbiról!
	
	\url{http://en.cppreference.com/w/cpp/algorithm/remove}
	\begin{lstlisting}
#include <iostream>
#include <vector>
#include <algorithm>

int main()
{
	std::vector<int> v{1,2,3,3,4,5,6};
	std::cout << v.size() << std::endl;
	std::remove(v.begin(), v.end(), 3);
	std::cout << v.size() << std::endl;
}
	\end{lstlisting}
	Legnagyobb meglepetésünkre, ez semmit se fog törölni: az \texttt{std::remove} átrendezi a konténert, úgy hogy a konténer elején legyenek a nem törlendő elemeket, és utána a törlendőek, és az első törlendő elemre visszaad egy iterátort. Ennek segítségével tudjuk, hogy ettől az iterátortól a \textit{past-the-end iterator}-ig minden törlendő.
	\begin{lstlisting}
std::vector<int> v{1,2,3,3,4,5,6};
std::cout << v.size() << std::endl;
auto it = std::remove(v.begin(), v.end(), 3);
v.erase(it, v.end());
std::cout << v.size() << std::endl;
	\end{lstlisting}
	\begin{note}
		Ebben a kódban van pár c++11-es újítás is: az \texttt{auto} kulcsszó megadható konkrét típus helyett. Ilyenkor az egyenlőségjel bal oldalán lévő objektum típusára helyettesítődik a kulcsszó. Példaképp, fent a fordító meg tudja határozni, hogy az \texttt{std::remove}-nak a visszatérési értéke itt \texttt{std::vector<int>::iterator} lesz, így a kód azzal ekvivalens, mintha ezt írtuk volna:
		
		{\centering\texttt{std::vector<int>::iterator it = std::remove(v.begin(), v.end(), 3);} \par}
		
		Bár az \texttt{auto} kulcsszót sokáig lehetne még boncolgatni, legyen annyi elég egyenlőre, hogy a template paraméter dedukciós szabályok szerint működik: azaz a const/volatile tulajdonságok \textit{(c/v qualifiers)} elvesznek közben. (c++11et nem szükséges tudni a vizsgához)
	\end{note}
	
	\medskip
	Megjegyzendő, hogy nem minden algoritmus működik minden iterátor típussal, és vannak egyes algoritmusokank speciális előfeltétele: például az \texttt{std::unique} egy rendezett konténert vár. Megjegyzendő még, hogy két különböző rendező algoritmusunk is van: \texttt{std::sort} és \texttt{std::stable\_sort}. AZ utóbbi garantálja, hogy az ekvivalens elemek relatív sorrendje ne vltozzon (pl. korábbi példában az \texttt{ADA} és a \texttt{C++}, melyek ekvivalensek, de nem egyenlőek).
	
	Fontos az is, hogy ezek az algoritmusok általában nagyon hatékonyak, így gyakran nem is érdemes sajt magunktól megírni.
	
	\medskip
	Fent kiírtam hogy \texttt{std::remove}, de ebből a namespace-t elhagyhatom.
	\begin{lstlisting}
//...
auto it = remove(v.begin(), v.end(), 3);
//...
	\end{lstlisting}
	Itt a fordító a függvény paraméterekből ki tudja találni (\textit{Argument Dependent Lookup, ADL}), hogy milyen névtérből van az algoritmus. Mivel a \texttt{vector} ugyanúgy az \texttt{std} névtérből származik, és azt adtunk meg paraméternek, tudni fogja a fordító, hogy a \texttt{remove}-ot is abban a névtérben kell keresni.
	
	\bigskip
	
	Vegyünk egy konténert, és jelöljünk meg benne bár elemet, melyektől azt szeretnénk, hogy legyenek egymás mellett, de minden más elem sorrendje ne változzon.
	\begin{center}
		\begin{tabular}{|c|c|c|c|c|c|c|c|}
			\hline
			&&*&&*&*&&*\\
			\hline
		\end{tabular}
		
		$\Downarrow$
		
		\begin{tabular}{|c|c|c|c|c|c|c|c|}
			\hline
			&&*&*&*&*&&\\
			\hline
		\end{tabular}
	\end{center}
	Próbáljuk a csillaggal jelölt elemet egymás mellé tenni: ezt megtehetjük pl. két \texttt{stable\_partition}-nel, mert az ekvivalens elemek sorrendjét nem változtatják. 
	
	Most csináljuk azt, hogy a kijelölt elemek a konténer végén legyenek, de a többi elem sorrendje ne változzon! Itt használhatjuk az előbbi algoritmusok után az \texttt{std::rotate}-et.
	\begin{center}
		
		\begin{tabular}{|c|c|c|c|c|c|c|c|}
			\hline
			&&*&*&*&*&&\\
			\hline
		\end{tabular}
		
		$\Downarrow$
		
		\begin{tabular}{|c|c|c|c|c|c|c|c|}
			\hline
			&&&&*&*&*&*\\
			\hline
		\end{tabular}
	\end{center}
	A legtöbb algoritmusnál, mely visszatér egy iterátorral, úgy jelzik, hogy az adott elemet nem találtak meg (stb), hogy egy \texttt{past-the-end} iterátort adnak vissza.
	%TODO jobb struktúrálás.
	\begin{center}
		\textit{,,Általában az emberek nem szeretik, hogyha kicsesznek velük, és ha kicseszel a kollégáiddal, morcosak lesznek''}
		
		/Horváth Gábor/
	\end{center}
\end{document}
