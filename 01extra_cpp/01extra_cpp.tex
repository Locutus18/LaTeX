\documentclass[a4paper,11.5pt]{article}
\usepackage[textwidth=170mm, textheight=230mm, inner=20mm, top=20mm, bottom=30mm]{geometry}
\usepackage[normalem]{ulem}
\usepackage[utf8]{inputenc}
\usepackage[T1]{fontenc}
\PassOptionsToPackage{defaults=hu-min}{magyar.ldf}
\usepackage[magyar]{babel}
\usepackage{amsmath, xcolor, amsthm,amssymb,paralist,array, ellipsis, graphicx}
%\usepackage{marvosym}

\usepackage{listings}
\lstset{
	language=C++, 
	basicstyle=\ttfamily, 
	keywordstyle=\color{blue}\ttfamily, 
	stringstyle=\color{red}\ttfamily,
	tabsize = 4
}

\makeatletter
\renewcommand*{\mathellipsis}{%
	\mathinner{%
		\kern\ellipsisbeforegap%
		{\ldotp}\kern\ellipsisgap%
		{\ldotp}\kern\ellipsisgap%
		{\ldotp}\kern\ellipsisaftergap%
	}%
}
\renewcommand*{\dotsb@}{%
	\mathinner{%
		\kern\ellipsisbeforegap%
		{\cdotp}\kern\ellipsisgap%
		{\cdotp}\kern\ellipsisgap%
		{\cdotp}\kern\ellipsisaftergap%
	}%
}
\renewcommand*{\@cdots}{%
	\mathinner{%
		\kern\ellipsisbeforegap%
		{\cdotp}\kern\ellipsisgap%
		{\cdotp}\kern\ellipsisgap%
		{\cdotp}\kern\ellipsisaftergap%
	}%
}
\renewcommand*{\ellipsis@default}{%
	\ellipsis@before
	\kern\ellipsisbeforegap
	.\kern\ellipsisgap
	.\kern\ellipsisgap
	.\kern\ellipsisgap
	\ellipsis@after\relax}
\renewcommand*{\ellipsis@centered}{%
	\ellipsis@before
	\kern\ellipsisbeforegap
	.\kern\ellipsisgap
	.\kern\ellipsisgap
	.\kern\ellipsisaftergap
	\ellipsis@after\relax}
\AtBeginDocument{%
	\DeclareRobustCommand*{\dots}{%
		\ifmmode\@xp\mdots@\else\@xp\textellipsis\fi}}
\def\ellipsisgap{.1em}
\def\ellipsisbeforegap{.05em}
\def\ellipsisaftergap{.05em}
\makeatother

\usepackage{hyperref}
\hypersetup{
	colorlinks = true	
}

\begin{document}
	%%%%%%%%%%%RÖVIDÍTÉSEK%%%%%%%%%%
	\setlength\parindent{0pt}
	\def\s{\hspace{0.2mm}\vphantom{\beta}}
	\def\Z{\mathbb{Z}}
	\def\Q{\mathbb{Q}}
	\def\R{\mathbb{R}}
	\def\C{\mathbb{C}}
	\def\N{\mathbb{N}}
	\def\Ra{\overline{\mathbb{R}}}
	
	\def\sume{\displaystyle\sum_{n=1}^{+\infty}}
	\def\sumn{\displaystyle\sum_{n=0}^{+\infty}}
	
	\def\narrow{\underset{n\rightarrow+\infty}{\longrightarrow}}
	\def\limn{\displaystyle\lim_{n\to +\infty}}
	\def\limx{\displaystyle\lim_{x\to +\infty}}
	
	\theoremstyle{definition}
	\newtheorem{theorem}{Tétel}[subsection] 
	
	\theoremstyle{definition}
	\newtheorem{definition}[theorem]{Definíció} 
	\newtheorem{example}[theorem]{Példa} 
	\newtheorem{task}[theorem]{Feladat} 
	\newtheorem{note}[theorem]{Megjegyzés}
	%%%%%%%%%%%%%%%%%%%%%%%%%%%%%%%%%%%%%%%%%%%%%%%%%%%%%%%%%%%%%%%%%%%%%
	\begin{center}
		{\LARGE\textbf{C++}}
		
		{\Large Gyakorlat jegyzet}
		
		$+/-$ megoldások
	\end{center}A jegyzetet \textsc{Umann} Kristóf készítette \textsc{Horváth} Gábor gyakorlatán. (\today)
	\section{Első $+/-$ megoldás}
	Írjunk egy programot, mely a vezetéknevünk első magánhangzóját kicseréli az utolsó mássalhangzóra, \texttt{[]} operátor használata nélkül!
	\begin{lstlisting}
char *p = "Husi";
tmp = *(p + 1);
*(p + 1) = *(p + 2);
*(p + 2) = tmp;
	\end{lstlisting}
	Le fog-e fordulni az alábbi kód?
	\begin{lstlisting}
int a;
const int *p = &a;
	\end{lstlisting}
	Igen, le fog, mert konstans pointerrel mutathatunk nem konstans változóra, ez nem sérti a konstans korrentséget.
	
	\medskip
	Mi az a szekvenciapont? Olyan nyelvi egységek, melyek előtt futási időben minden kifejezésnek ki kell értékelődnia (pl. pontosvessző).
	
	\medskip
	Mi lesz az alábbi kód kimenete?
	\begin{lstlisting}
int t[][2] = {{2,3}, {4,5}, {9, 0}};
std::cout << *(0 + *(t + 1));
std::cout << 2[*(t + 3)];
std::cout << *(2 + 1[t]);
	\end{lstlisting}
	Rendre 4, nem definiált (túlindexelünk), és 9.
	\section{Második $+/-$ megoldás}
	Paraméter dedukció az, amikor a fordító ki tudja találni a template paraméterét egy függvénynek (nagyon dióhéjban). Példa:
	\begin{lstlisting}
template <class T>
void f(const T& param) {}

int main()
{
	f<std::string>("Hello"); //nincs dedukcio
	f("Hello"); //van dedukcio
}
	\end{lstlisting}
	
	A második feladat egy olyan template osztály létrehozása, amely heapen tárol egy objektumot, és reguláris típus. Mivel ez egy template osztály, nem tudjuk milyen típust kapunk, alapértelmezetten vegyük úgy, hogy a másolás költséges, így másolás helyett vegyünk át konstans referenciákat.
	\begin{lstlisting}
template <class T>
class X
{
public:
	X(const T& val) : data(new T(val)) {}
	~X() {delete data;}
	X(const X &other) : data(new T(*other.data)) {}
	X &operator=(const X &other)
	{
		if (this == &other)
			return *this;
		delete data;
		data = new T(*other.data);
		return *this;
	}
private:
	T *data;
};

int main()
{
	X<int> a;
	X<int> y(a);
	y = a;
}
	\end{lstlisting}
	Bővebb magyarázathoz az 5-6. gyakorlat anyagát érdemes megnézni.
	
	\medskip
	Vizsgán gyakran (DE NEM MINDIG!) nem kell megírni az értékadó operátort meg copy konstruktort, stb, mert gyakran olyan adattagokat kell tárolnunk, mint az \texttt{std::vector}. Mivel az alapértlemezett copy konstruktor és értékadó operátor az adattagok copy konstruktorát és értékadó operátorát hívja meg, így reguláris típusokat tárolunk, akkor a mi típusunk is reguláris lesz.
	\section{Harmadik $+/-$ megoldás}
	Helyes-e az elábbi kód?
		
	\begin{lstlisting}
struct X
{
	virtual a();
};

struct Y : public X {/* ... */};

int main()
{
	X *x = new Y;
	delete x;
}
	\end{lstlisting}
		
	A válasz az hogy nem, mert itt egy \texttt{Y} típusú objektumot polimorfikusan használunk, és amikor egy \texttt{X} típusú objektummal hivatkozunk rá, törléskor felmerül a kérdés: Most a dinamikus vagy a statikus destruktor szerint töröljünk? Első esetben a fordító gondoskodna arról, hogy az öröklődési ágnak megfelelően lefussanak a destruktorok, azonban mivel ebben az esetben \texttt{X} destruktora \textbf{nem virtuális}, így a statikus destruktor hívódik meg. 
	
	Ez egy mondatban azt jeleni, hogy egy \texttt{Y} típusú objektumra \texttt{X} destrukora hívódik meg, ami nem definiált viselkedés.
	\bigskip
	
	Soroljuk fel a standard szekvenciális konténereket! \texttt{std::vector, std::list, std::deque}
	\bigskip
	
	 Írjunk egy funktort mely két string hosszát hasonlítja össze!
	\begin{lstlisting}
struct PlusMinusFunctor
{
	bool operator()(const std::string &lhs, const std::string &rhs) const
	{
		return lhs.size() < rhs.size();
	}
};
		\end{lstlisting}
\end{document}
