\documentclass[a4paper,11.5pt,table]{article}
\usepackage[textwidth=170mm, textheight=230mm, inner=20mm, top=20mm, bottom=30mm]{geometry}
\usepackage[normalem]{ulem}
\usepackage[utf8]{inputenc}
\usepackage[T1]{fontenc}
\PassOptionsToPackage{defaults=hu-min}{magyar.ldf}
\usepackage[magyar]{babel}
\usepackage{amsmath, amsthm,amssymb, paralist, tikz, multirow}
\usetikzlibrary{arrows, positioning}

\usepackage{listings}
\lstdefinestyle{customc}{
	belowcaptionskip=1\baselineskip,
	breaklines=true,
	frame=L,
	xleftmargin=\parindent,
	language=C,
	showstringspaces=false,
	basicstyle=\footnotesize\ttfamily,
	keywordstyle=\bfseries\color{green!40!black},
	commentstyle=\itshape\color{purple!40!black},
	identifierstyle=\color{blue},
	stringstyle=\color{orange},
}

\lstdefinestyle{customasm}{
	belowcaptionskip=1\baselineskip,
	frame=L,
	xleftmargin=\parindent,
	language=[x86masm]Assembler,
	basicstyle=\footnotesize\ttfamily,
	commentstyle=\itshape\color{purple!40!black},
}

\lstset{escapechar=@,style=customc}

\usepackage{hyperref}

\begin{document}
	%%%%%%%%%%%RÖVIDÍTÉSEK%%%%%%%%%%
	\setlength\parindent{0pt}
	\def\<{<\hspace{0mm}<}
	
	\theoremstyle{definition}
	\newtheorem{note}{Megjegyzés}[subsection]
	%%%%%%%%%%%%%%%%%%%%%%%%%%%%%%%%%%%%%%%%%%%%%%%%%%%%%%%%%%%%%%%%%%%%%
	
	\begin{center}
		{\LARGE\textbf{C++}}
		
		{\Large Gyakorlat jegyzet}
		
		11. óra.
	\end{center}
	A jegyzetet \textsc{Umann} Kristóf készítette \textsc{Brunner} Tibor gyakorlatán. (\today)
	\section{Objektum orientált programozás}
	Objektum orientált programozásnál e hármat várjuk el:
	\begin{itemize}
		\item enkapszuláció
		\item kód újrafelhasználás
		\item adat elrejtés
	\end{itemize}
	Az, hogy egy adott nyelv ezt hogyan valósítja meg, az a saját maga dolga. Vannak nyelvek, melyek jobban alkalmasan objektum orientált programozásra, mint mások. Példaképp, a C nyelvre is mondhatjuk, hogy objektum orientált, bár nincs benne örökölődés, és nem igazán kézenfekvő az adatok elrejtése sem (bár lehetséges). Nyilvánvalóan ennél egyel praktikusabb nyelvek is léteznek, példaképp Java, és nyilván a c++ is, melyek sokkal erősebb nyelvi eszközökkel rendelkeznek ezek megvalósítására.
	
	Fontos, hogy az objektum orientáltság fogalmát ne csak az osztályokhoz kössük!
	
	\subsection{Öröklődés (\textit{inheritance})}
	Tekintsük az alábbi kódot:
	\begin{lstlisting}
class Sikidom
{
	int x;
	int y;
public:
	Sikidom(int x, int y) : x(x), y(y) {}
	double terulet() const
	{
		//...
	}
};
	\end{lstlisting}
	Itt a célunk egy általános síkidom megadása, azonban bajban vagyunk, hisz nem tudjuk megmondani, hogy épp egy tetszőleges síkidomnak milyen lesz a területe. Hozzunk létre egy kört, hisz annak a területét meg tudjuk adni! Próbáljuk ehhez használni az öröklődést: vegyünk át mindent, ami a Síkidomban van, elvégre minden kör egy síkidom!
	
	\begin{lstlisting}
class Kor : public Sikidom
{
	double r;
public:
	double terulet() const
	{
		return r * r * 3.14;
	}
};
	\end{lstlisting}
	Az öröklődés következtében \texttt{Kor} osztály \texttt{Sikidom} összes adattagját és metódusát átveszi. Fentebb látható az öröklődés szintaktikája: kettőspont, és utána egy öröklődési típus (ez esetben \texttt{public}, erről nemsoká lesz bővebben szó). Mondhatjuk azt is, hogy a \texttt{Kor} \textit{az egy} \texttt{Sikidom}, hisz minden, a \texttt{Sikidom}ra jellemző tulajdonsággal rendelkezik.
	\begin{note}
		Nagyon fontos, hogy a 3,14-nél pontot és nem vesszőt használunk. Ha ezt írnánk:
		\begin{lstlisting}
int i = 1, 2, f(), g();
		\end{lstlisting}
		
		Akkor rendre a vessző paraméter előtti tag mindig kiértékelődik, és utána értékeli ki a következőt, a korábbi eredményt meg elhajítja. A végén az egész azzal lesz ekvivalens, mintha simán azt írnánk, hogy \texttt{int i = g();}.
		
		Egy fél mondat erejéig jegyezzük meg azt is, hogy lebegőpontos számokat sose hasonlítsunk össze! Jusson eszünkbe, ami Numerikus Módszereken hangzik el: minden lebegőpontos szám tartalmazhat egy kis pontatlanságot.
		\begin{lstlisting}
for (double d = 0; d != 1; d += 0.1)
{
	std::cout << d << ' ';
}
		\end{lstlisting}
		Itt bár azt várnánk, hogy a kimenet \texttt{0 0.1 0.2 ... 1.0} legyen, legnagyobb valószínűséggel végtelen ciklusba futunk. Ez amiatt van, hogy a 0.1 (várhatóan) tartalmaz egy kis pontatlanságot, és így hiába írja ki  a programunk hogy \texttt{d} értéke 1, az várhatóan csak nagyon közel lesz hozzá.
	\end{note}
	Az, hogy publikusan örököltünk (erre volt a \texttt{public} kulcsszó), azt jelenti, hogy egy {altípus}-t (\textit{subtype}) hoztunk létre. 
	Ezt azt is jelenti, hogy minden helyre, ahol \texttt{Sikidom}-ot szeretnénk használni, \texttt{Kor}-t is használhatunk.
	
	Van egy apróbb probléma a fenti kóddal azonban: Az osztály azon adattagjai, melyek \texttt{private}-ként vannak deklarálva, csak az adott számára elérhetőek.
	
	Ez probléma hisz mi \texttt{Kor}-t azért hoztuk létre, hogy speciálisabb feladatokat tudjuk végrehajtani \texttt{Sikidom} tulajdonságaival, azonban \texttt{x} és \texttt{y} privát, és csak \texttt{Sikidom} tud hozzáférni. Írjuk át ezt protectedre:
\begin{lstlisting}
class Sikidom
{
protected:
	int x;
	int y;
public:
	Sikidom(int x, int y) : x(x), y(y) {}
	double terulet() const {}
};
\end{lstlisting}
	A \texttt{protected} tagokhoz csak az adott osztály fér hozzá, és azon osztályok, melyek ebből örökölnek. Nem csak publikusan tudunk örökölni, \texttt{protected}-del, és \texttt{private}-el is. Az alábbi táblázat mutatja, hogy az származott osztály (\textit{derived class}) milyen adattagokhoz fér hozzá a bázisosztályból (\textit{base class}) különböző öröklődési típusok esetén:
	\begin{center}
		\setlength{\extrarowheight}{2pt}
		\begin{tabular}{|c||c|c|c|}
			\hline
			Öröklődés típusa:&public&protected&private\\
			\hline
			\hline
			Saját adatok/metódusok& \cellcolor{green!20}igen&\cellcolor{green!20}igen&\cellcolor{green!20}igen\\
			\hline
			Örökölt adatok/metódusok (public)&\cellcolor{green!20}igen&\cellcolor{green!20}igen&\cellcolor{green!20}igen\\
			\hline
			Örökölt adatok/metódusok (protected)&\cellcolor{green!20}igen&\cellcolor{green!20}igen&\cellcolor{green!20}igen\\
			\hline
			Örökölt adatok/metódusok (private)&\cellcolor{red!20}nem&\cellcolor{red!20}nem&\cellcolor{red!20}nem\\
			\hline
		\end{tabular}
	\end{center}
	Ez pedig mutatja egy adott adattag/metódus védettségét öröklődés után:
	\begin{center}
		\begin{tabular}{|c||c|c|c|}
			\hline
			Öröklődés típusa:&public&protected&private\\
			\hline
			\hline
			Örökölt adatok/metódusok (public)&\cellcolor{green!20}public&\cellcolor{orange!20}protected&\cellcolor{red!20}private\\
			\hline
			Örökölt adatok/metódusok (protected)&\cellcolor{orange!20}protected&\cellcolor{orange!20}protected&\cellcolor{red!20}private\\
			\hline
			Örökölt adatok/metódusok (private)&\cellcolor{red!20}private&\cellcolor{red!20}private&\cellcolor{red!20}private\\
			\hline
		\end{tabular}
	\end{center}
	Az öröklődés típusa ha \texttt{class}ból öröklünk \texttt{private}, ha \texttt{struct}ból akkor \texttt{public.}
	\begin{note}
		A publikus öröklődést szokás {az egy} (\textit{is a}), privátot \texttt{van egy} (\textit{has a}) kapcsolatnak is hívni.
	\end{note}
	Picit térjünk visza a fenti értékadáshoz, mert ott azért van még mit megbeszélni.
	\begin{lstlisting}
Kor k;
Sikidom s;
s = k; //ok
	\end{lstlisting}
	Azt már letisztáztuk, hogy \texttt{Kor} \texttt{Sikidom} altípusa, de azonban ez akkor is két különbötő típus, hogyan lehet, hogy nem kapunk fordítási idejű hibát?
	
	A válasz az, hogy a fordító meg tudja oldani e két típus közötti értékadást. Mivel \texttt{Kor}-ben benne van maga \texttt{Sikidom} is, ezért meghívja az ahhoz tartozó értékadó operátort, és a \texttt{Kor}-ben lévő \texttt{Sikidom}-ot adja majd értékül.
	
	Ez azonban magával hordozza azt is, hogy minden információ, ami \texttt{Kor}-respecifikus (pl. sugár (\texttt{r})) elveszik. Ez az un. \textit{slicing}. Ebben az esetben, lévén a fordító generálta nekünk az értékadó operátort, és az alapértelmezetten minden adattagra meghívja az arra definiált értékadó operátort, gyakorlatilag ez fog történni:
	\begin{lstlisting}
s.x = k.x;
s.y = k.y;
	\end{lstlisting}
	
	Ezt a \textit{slicing} dolgot meg tudjuk kerülni, ha egy pointert hozunk létre: Hisz egy \texttt{Sikidom}-ra mutató pointer mutathat \texttt{Kor}-re is, és mivel az ''csak'' mutogat, nem fog levágni semmit (referenciák szintén!). 
	\begin{lstlisting}
Kor k;
Sikidom *sp = &k; 
Sikidom &sr = k;
	\end{lstlisting}
	Így minden adathoz, ami \texttt{Kor}-ben \texttt{Sikidom}-tól lett örökölve, hozzá tudunk férni \texttt{sp}-n keresztül. Fontos megjegyeznünk azonban, hogy egy \texttt{Sikidom} típusú pointer csak ''\texttt{Sikidom}nyi'' adattagokat/metódusokat tud kezelni.
	
	Mi fog történni, ha meghívjuk a \texttt{terulet} metódust?
	\begin{lstlisting}
Kor k;
Sikidom *sp = &k;
sp->terulet();
	\end{lstlisting}
	Itt azt fogjuk tapasztalni, hogy \texttt{Sikidom} függvénye fog meghívódni. Ennek oka az, hogy \texttt{sp} statikus típusa (\textit{static type}) az \texttt{Sikidom}, és hatékonyság végett alapértelmezetten mindig a statikus típushoz tartozó függvényt hívódik meg. Egy pointernek azonban van un. dinamikus típusa is (\textit{dynamic type}), ami ebben az esetben \texttt{Kor} lesz. A dinamikus típust csak futási időben lehet ismert, hisz lehetetlen azt megállapítani fordítási időben, hogy milyen típusú objektumra fog mutatni az \texttt{sp}. Ha mindig a dinamikus típus szerint hívná meg a program függvényeket, annak nagy futási idejű költsége lenne. Ahhoz, hogy rákényszerítsük a fordítót arra, hogy a dinamikus típus szerint fordítson, használnunk kell a \texttt{virtual} kulcsszót.
	%TODO miért szar a láncolt lista: túl költséges a processzorba pakolás ...
	\subsection{Virtuális függvények}
	Azokat az osztályokat, melyek legalább egy virtuális függvénnyel rendelkeznek, polimorfikus osztálynak (\textit{polymorphic class}) nevezzük.
\begin{lstlisting}
class Sikidom
{
protected:
	int x;
	int y;
public:
	Sikidom(int x, int y) : x(x), y(y) {}
	virtual double terulet() const //virtualis!
	{
		//...
	}
};

class Kor : public Sikidom
{
	double r;
public:
	double terulet() const
	{
		return r * r * 3.14;
	}
};
\end{lstlisting}
	Picit bánthat minket az, hogy hogyan jöhet elő ez a probléma, ugyanis úgy tűnik, mintha sértenénk a \textit{one definition rule}-t, hisz létrehoztunk két függvényt azonos visszatérési értékkel és paraméterlistával: a kiút az, hogy itt felüldefiniálás (\textit{override}) fog bekövetkezni, azaz a \texttt{Kor}-béli \texttt{terulet} le fogja árnyékolni a \texttt{Sikidom} \texttt{terulet} függvényét. Viszont ez a leárnyékolás nem ér sokat, ha nincs \texttt{virtual} kulcsszó, és ha egy \texttt{Sikidom} típusú pointerrel akarjuk ezt használni.
	
	A bázisosztály függvényének leárnyékolása azonban nem azt jelenti, hogy azt a függvényt nem lehet meghívni. Ha explicit módon jelezzük a fordítónak, hogy a bázisosztályban definiált metódust szeretnénk meghívni, akkor minden virtualitás ellenére is az lesz meghívva:
	\begin{lstlisting}
int main()
{
	Kor k;
	std::cout << k.Sikidom::terulet();
}
	\end{lstlisting}
	Ezt explicit névfeloldásnak (\textit{explicit scope resolution}) nevezzük.
	\begin{note}
		Minden metódus, mely öröklődésnél a bázisosztály egy virtuális metódusával megegyező visszatérési értékkel, paraméterlistával és konstanssággla rendelkezik, implicit módon virtuális lesz. (pl.: \texttt{Kor}-ben \texttt{terulet} implicit módon virtuális lesz, mert \texttt{Sikidom}-ban is az volt)
	\end{note}
	Konstruktor sose lehet virtuális.
	\subsection{Tisztán virtuális függvények}
	Bár ha háromszögekre felvágunk egy síkidomot, meg tudjuk mondani a területét, azért látjuk, hogy iszonyatos szenvedés nélkül egy általános síkidom területének megadása aligha lehetséges. Ezért gyakran úgy dönthetünk, hogy egy adott függvénynek csak a felületét (\textit{interface}) örököltetjük, de implementációját nem.
	
	A fenti osztálynál azonban probléma, hogyha valaki létrehoz egy \texttt{Sikidom} típusú objektumot, és meghívja a \texttt{terulet} függvényt, jogosan várja, hogy az ki is számítsa neki. Ennek megkerülésére alkalmazahtunk egy trükköt: \texttt{Sikidom}ban \texttt{terulet}-et tisztán virtuálissá tesszük:
	
\begin{lstlisting}
class Sikidom
{
protected:
	int x;
	int y;
public:
	Sikidom(int x, int y) : x(x), y(y) {}
	virtual double terulet() const = 0;
};
\end{lstlisting}
	Mostmár \texttt{terulet} tisztán virtuális (\textit{pure virtual}) \texttt{Sikidom}-ban. Az olyan osztályokat, melyekből nem lehet objektumot létrehozni, absztraktnak nevezzük. Az olyan osztályokból, melyek tartalmaznak tisztán virtuális metódusokat, nem tudunk objektumot létrehozni.
	\begin{note}
		Egy osztály úgy is lehet absztrakt, hogy \texttt{protected}-dé tesszük a konstruktorait (erről majd később). Az egyszerűség kedvéért, ha absztrakt osztályról beszélünk, gondoljunk egy tisztán virtuális metódussal rendelkező osztályra.
	\end{note}
	Egy olyan osztály, mely egy absztrakt osztályból örököl, ahhoz, hogy példányosítható legyen (lehessen olyan típusú objektumokat létrehozni), felül kell írnia az összes abban található tisztán virtuális függvényt.
	\begin{lstlisting}
int main()
{
	Sikidom s; //nem ok, forditasi ideju hiba
	Kor k; //ok
	Sikidom *sp = &k;
	Sikidom &sr = k;
	
	sp->terulet();
	sr.terulet();
}
	\end{lstlisting}
	Mint láthatjuk, absztrakt típusú pointert és referenciát lehet létrehozni, így a polimorfizmus összes előnyével élhetünk, továbbá mindenkit sikerült rákényszeríteni a \texttt{terulet} függvény definiálására, aki \texttt{Sikidom}-ból örököl.
	
	Persze, ettől mi még írhatunk definíciót egy tisztán virtuális függvényhez:
\begin{lstlisting}
class Sikidom
{
protected:
	int x;
	int y;
public:
	Sikidom(int x, int y) : x(x), y(y) {}
	virtual double terulet() const = 0;
};

double Sikidom::terulet()
{
	return 0;
}
\end{lstlisting}
	Figyeljük meg, hogy egy tisztán virtuális függvényt csak a deklarációtól külön tudunk definiálni. 
	
	Ez a trükk például akkor hasznosítható, ha egy van egy absztrakt bázisosztályunk, melynek szeretnénk, hogy öröklés után egy függvényét mindenki definiáljon felül, de akarunk adni hozzá egy alap implementációt.
\begin{lstlisting}
class Kor : public Sikidom
{
	double r;
	public:
	double terulet() const
	{
		return Sikidom::terulet();
	}
};
\end{lstlisting}
	
	A későbbiekben tekintsünk \texttt{Sikidom}-ra úgy, mintha nem lenne tisztán virtuális függvénye.
	\subsection{Kód módosítása polimorfizmusnál}
	\smallskip
	Szánjunk pár szót meglévő kód módosításáról, ha polimorfizmusról is van szó.
	\begin{lstlisting}
void f(Kor* k)
{
	//...
}
	\end{lstlisting}
	Lehet, az implementáció egy pontján úgy döntünk, hogy ez várjon inkább síkidomokat.
	\begin{lstlisting}
void f(Sikidom* k)
{
	//...
}
	\end{lstlisting}
	Ezzel semmi gond nem lesz, ugyanis ha mi \texttt{f}-nek továbbra is \texttt{Kor}öket fogunk adni, egy \texttt{Sikidom} típusú pointer tud még mutatni \texttt{Kor} típusra. Ennek a tanulsága az, hogy paramétereknél \textbf{felfelé} az öröklődési láncon lehet haladni, lefelé nem, hiszen egy síkidomra mutató pointer mutathat körre, de fordítva nem. Továbbá, ha visszatérési értéknél:
	\begin{lstlisting}
Sikidom f(...) {}
	\end{lstlisting}
	úgy döntünk, inkább kört adunk vissza:
	\begin{lstlisting}
Kor f(...) {}
	\end{lstlisting}
	nem lesz semmi gond, hisz itt is, minden kör \textit{az egy} síkidom. Azaz ha módosítjuk egy függvény visszatérési értékét, \textbf{lefelé} haladjunk az öröklődési ágon.
	\begin{lstlisting}
Kor *k;
Sikidom * sp = f(k);
	\end{lstlisting}
	Bajban lennénk, ha \texttt{Sikidom}-nál egy általánosabb osztályt adna vissza \texttt{f}.
	\medskip
	
	Legyen adott egy osztályhierarchia, melyben \texttt{B} publikusan örököl \texttt{A}-ból, \texttt{C} osztály \texttt{B}-ből, és végül \texttt{D} osztály \texttt{C}-ből. Ilyenkor \texttt{A} a ,,legáltalánosabb'' osztály, és \texttt{D} a ,,legspeciálisabb''.
	
	Az alábbi táblázat jól összefoglalja az itt megállapítottakat (piros: a módosítás nem megfelelő, zöld: a módosítás jó lesz):
	\begin{center}
		\setlength{\extrarowheight}{2pt}
		\begin{tabular}{|l||c|c|c|c|}
			\hline
			&Eredeti típus&\multicolumn{3}{c|}{Módosított típus}\\
			\hline
			\hline
			\multirow{4}{*}{Paraméterek}&A*&\cellcolor{red!20}B*&\cellcolor{red!20}C*&\cellcolor{red!20}D*\\
			\cline{2-5}
			&B*&\cellcolor{green!20}A*&\cellcolor{red!20}C*&\cellcolor{red!20}D*\\
			\cline{2-5}
			&C*&\cellcolor{green!20}A*&\cellcolor{green!20}B*&\cellcolor{red!20}D*\\
			\cline{2-5}
			&D*&\cellcolor{green!20}A*&\cellcolor{green!20}B*&\cellcolor{green!20}C*\\
			\hline
			\hline
			\multirow{4}{*}{Visszatérési érték}&A*&\cellcolor{green!20}B*&\cellcolor{green!20}C*&\cellcolor{green!20}D*\\
			\cline{2-5}
			&B*&\cellcolor{red!20}A*&\cellcolor{green!20}C*&\cellcolor{green!20}D*\\
			\cline{2-5}
			&C*&\cellcolor{red!20}A*&\cellcolor{red!20}B*&\cellcolor{green!20}D*\\
			\cline{2-5}
			&D*&\cellcolor{red!20}A*&\cellcolor{red!20}B*&\cellcolor{red!20}C*\\
			\hline
		\end{tabular}
	\end{center}
	\subsection{Többszörös öröklődés}
	C++ban lehetőségünk van többszörösen örökölni. Ez azt jelenti, hogy egyszerre több osztályból is öröklünk.
	\begin{lstlisting}
class BaseOne {};
class BaseTwo {};

class Derived : public BaseOne, public BaseTwo {};
	\end{lstlisting}
	Általában igaz az, hogy a többszörös öröklődésnél több a galiba mint a haszon, így hacsak nincs nagyon jó okunk rá, ne erőltessük. Nehezen lesz velük átlátható, hogy milyen lesz egy pointer dinamikus típusa, rászorulhatunk side cast-okra (erről később), és egyéb nem elegáns módszerekre.
	\begin{lstlisting}
class Negyszog 
{
protected:
	double x;
public:
	virtual double terulet() = 0;
};

class Rombusz : public Negyszog
{
public:
	double terulet() = 0; 
	//nem jut eszembe a keplet, igy legyen ez is tisztan virtualis.
};

class Deltoid : public Negyszog
{
public:
	double terulet() = 0;
};

class Negyzet : public Deltoid, public Rombusz
{
public:
	double terulet()
	{
		return x * x;
	}
};
	\end{lstlisting}
	A legutóbbi osztály természetesnek tűnhet, hisz egy négyzet deltoid is, meg rombusz is, így szeretnénk, ha minden függvény, ami \texttt{Deltoid}-ot vagy \texttt{Rombusz}-t vár, \texttt{Negyzet}-et is elfogadjon. Azonban a fenti kód nem fordul le, hisz mind \texttt{Rombusz}-ban, mind \texttt{Deltoid}-ban van egy teljes \texttt{Negyszog} is. Így \texttt{Negyzet} kétszer fogja tartalmazni annak összes adattagját, így \texttt{x}-et is, és a fordító nem tudja kitalálni, mi épp melyikre gondolunk. 2 kézenfekvő megoldás lehet:
	\begin{lstlisting}
class Negyzet : public Deltoid, public Rombusz
{
public:
	double terulet()
	{
		return Deltoid::x * Deltoid::x;
	}
};
	\end{lstlisting}
	Azonban egyel szerencsésebb lenne, ha csak 1 darab \texttt{Negyszog} szerepelne \texttt{Negyzet}-ben. Erre megoldás lehet a virtuális öröklődés.
\begin{lstlisting}
class Negyszog 
{
protected:
	double x;
public:
	virtual double terulet() = 0;
};

class Rombusz : virtual public Negyszog //virtualis!
{
public:
	double terulet() = 0; 
};

class Deltoid : virtual public Negyszog
{
public:
	double terulet() = 0;
};

class Negyzet : public Deltoid, public Rombusz
{
public:
	double terulet()
	{
		return x * x;
	}
};
\end{lstlisting}
	Ezzel garantáltuk is ezt, de milyen áron? A virtuális öröklődésnek, csak úgy mint a virtuális metódusoknak futási idejű költsége van, így ha egy mód van rá, kerüljük.
	
	%TODO uml
	Annak ellenére, hogy a többszörös öröklődés erősen ellenjavallott dolog, van kivétel a standard könyvtáron belül is, mely él vele: az \texttt{iostream}. A beolvasó és kiírató stream-ek közös tulajdonságait tartalmazza \texttt{iosbase}, melyből örököl \texttt{istream} és \texttt{ostream} is, és mindkettőből az \texttt{iostream}.
	\subsection{Túlterhelés és felüldefiniálás}
	Próbáljunk megírni egy sítábor programot, melyben azt oldjuk meg, hogy a kétszobás ágyakban vagy 2 fiú, vagy 2 lány aludjon!
\begin{lstlisting}
#include <iostream>

class Sielo
{
public:
	virtual void szobatars(Sielo*)
	{
		std::cout << "sielo";
	}
};

class Lany : public Sielo
{
public:
	void szobatars(Sielo*)
	{
		std::cout << "valamilyen sielo";
	}
	void szobatars(Lany*)
	{
		std::cout << "lany sielo";
	}
};

class Fiu : public Sielo
{
public:
	void szobatars(Sielo*)
	{
		std::cout << "valamilyen sielo";
	}
	void szobatars(Fiu*)
	{
		std::cout << "fiu sielo";
	}
};

int main()
{
	Lany *lany = new Lany;
	Fiu *fiu = new Fiu;
	Sielo *sielo = lany;
	
	sielo->szobatars(fiu); //valamilyen sielo
	lany->szobatars(fiu); //valamilyen sielo
	lany->szobatars(lany); //lany sielo
	sielo->szobatars(lany); //valamilyen sielo
	
	delete lany; delete fiu;
}
\end{lstlisting}
	Figyeljük meg, hogy bár megöröklik mind \texttt{Fiu}, mind \texttt{Lany} a \texttt{szobatars(Sielo*)} függvényt, de azt még felül is írják (\textit{override}), ezen kívül még a függvényt túl is terhelik (\textit{overload}) rendre \texttt{szobatars(Fiu*)} és \texttt{szobatars(Lany*)} függvénnyel. 
	
	Feltűnhet, hogy sajnos ezzel a kóddal nem sikerül elérnünk, hogy a lányok és fiúk szeparálva maradjanak. Ennek oka az, hogy a harmadik eset kivételével felüldefiniálás következik be, és csupán a harmadik esetben lesz függvénytúlterhelés. Az, hogy ott miért nem felüldefiniálás következik be, a korábbi szekció magyarázza: a paraméter úgy változott, hogy az öröklődési ág szerint speciálisabb típusra váltottunk, mi meg megállapítottuk fent, hogy csak általánosabb típusra váltás eredményezhet továbbra is jól forduló kódot.
	
	A másik három esetben \texttt{fiu} is és \texttt{lany} is a \texttt{szobatars(Sielo*)} függvénybe fog menni.
	
	\medskip
	Ha kitörölnénk rendre mindkét leszármazott osztályból a \texttt{szobatars(Sielo*)} metódust, még mindig bajban lennénk, hisz egy \texttt{Sielo} típusú pointer dinamikus típusa lehet \texttt{Lany}, és minden probléma nélkül meg tudnánk adni neki paraméterül egy \texttt{Fiu} típusú objektumot, hisz minden \texttt{Fiu} \textit{az egy} \texttt{Sielo}.  Ez az eset annyival lenne jobb, hogy \texttt{lany} pointeren keresztül a \texttt{szobatars} függvénynek nem adhatjuk meg \texttt{fiu}-t paraméterül.
	
	\medskip
	Forrás: \url{http://bruntib.web.elte.hu/read.php?3,52} (\textsc{Brunner} Tibor honlapja)
	
	\subsection{Cast-ok}
	
	Néha abba a kellemetlen helyzetbe kerülhetünk, hogy van egy adott pointerünk/referenciák, és a dinamikus típusa \texttt{Kor}, azonban statikus típusa \texttt{Sikidom}, de mégis fontos, hogy \texttt{Kor}-re vonatkozó dolgokat használjunk. Ilyenkor segíthetnek a \textit{cast}ok:
	\begin{lstlisting}
void f(Sikidom *s)
{
	Kor *k = dynamic_cast<Kor*>(s); //korkent szerentenk hasznalni
}
	\end{lstlisting}
	Ebben a kódrészletben megpróbáljuk \texttt{s}-t átkonvertálni \texttt{Kor} pointer típusra. Megfigyelhető, hogy kacsacsőrbe írtuk azt a típust, amelyre konvertálunk, és utána gömbölyű zárójelbe azt, melyből konvertálnánk.
	
	A \texttt{dynamic\_cast} át tud konvertálni el pointert vagy referenciát egy másik típusú pointerre vagy refenciára. Ezt a kasztot csak polimorfikus osztályoknál használhatjuk. Leginkább 3 dologra jó, leszármazott típusról bázis típusra konvertálásra (\textit{upcast}), bázisról leszármazottra (\textit{downcast}), valamint többszörös öröklődésnél bázisok közötti váltásra (\textit{sidecast}) (és nyilván mind a 3 esetben pointerekkel vagy referenciákkal).
	
	A \texttt{dynamic\_cast} futási időben ellenőriz, és ha a cast nem sikeres, jelzi azt. Sikertelen lehet úgy, ha pl. ilyet csinálunk:
	\begin{lstlisting}
Sikidom *s = new Sikidom;
Kor *k = dynamic_cast<Kor*>(*s);
	\end{lstlisting}
	Ebben az esetben egy hibás downcast-ot csináltunk, hisz \texttt{s}-nek a dinamikus típusa is \texttt{Sikidom}. Mivel minden \texttt{Kor} \textit{az egy} \texttt{Sikidom}, de ez fordítva nem igaz, így a konverzió nem lehetséges. Ilyenkor a \texttt{dynamic\_cast} nullpointert ad vissza. Ha referenciákkal jutunk ilyen helyzetbe, akkor egy \texttt{std::bad\_cast} típusú kivételt fog dobni.
	\begin{lstlisting}
class Base 
{
	virtual int f(){} //polimorfizmus szukseges
};
class DerivedOne : virtual public Base {};
class DerivedTwo : virtual public Base {};
class DerivedLast : public DerivedOne, public DerivedTwo {};

int main()
{
	DerivedLast *dlp = new DerivedLast;
	Base *bp = dynamic_cast<Base*>(dlp); //upcast, dynamic_cast folosleges
	DerivedOne *dop = dynamic_cast<DerivedOne*>(bp); //downcast
	DerivedTwo *dtp = dynamic_cast<DerivedTwo*>(dop); //sidecast
	delete dlp;
	
	bp = new Base;
	dop = dynamic_cast<DerivedOne*>(bp);  //dop nullpointer
	delete bp;
	
	Base b;
	try
	{
		DerivedOne &dor = dynamic_cast<DerivedOne&>(b); 
		//std::bad_cast-ot fog dobni
	}
	catch (std::bad_cast &exc)
	{
		std::cout << exc.what() << std::endl;
	}
}
	\end{lstlisting}	
	\begin{note}
		A \texttt{dynamic\_cast} előtt nincs \texttt{std}. Ennek oka, az, hogy a \texttt{dynamic\_cast} egy operátor, és nem egy standard függvény.
	\end{note}
	
	Ez azonban nem hatékony, hisz a \texttt{dynamic\_cast} futási időben járja végig az öröklődési láncot. Ha nem vagyunk biztosak benne, hogy garantáltan lehetséges a castolás, akkor \texttt{dynamic\_cast}-ot érdemes használni, ha azonban biztosak vagyunk benne hogy az lehetséges, akkor \texttt{static\_cast} a jobb, mely fordítási időben végzi el ezt, hatékonyabb, azonban kevésbé biztonságos, ugyanis ha a cast sikertelen, akkor nem nullpointert ad vissza, és az így kapott pointert ha dereferáljuk, nem definiált viselkedés áldozatai leszünk.
	\subsection{Konstruktorok és destruktorok öröklődésnél}
	Konstruktorokról még beszélnünk kéne: a szemfüleseknek feltűnhetett, hogy a fenti kód le se fordul! (tekintsünk el attól, hogy a \texttt{terulet} metódus hiányzik)
	\begin{lstlisting}
class Sikidom
{
protected:
	int x, y;
public:
	Sikidom(int x, int y) : x(x), y(y) {}
};

class Kor : public Sikidom
{
	double r;
};
	\end{lstlisting}
	Mivel konstruktort nem írtunk \texttt{Kor}-nek, így a fordító generál nekünk egyet, melyben megpróbálja \texttt{Sikidom}-hoz tartozó adattagokat \texttt{Sikidom} paraméter nélküli konstruktorával meghívni. Azonban mivel nincs neki ilyen, fordítási idejű hibát kapunk. Ilyenkor csak egy lehetőségünk van \texttt{Sikidom} módosítása nélkül, írnunk kell \texttt{Kor}-nek egy konstruktort, melynek inicializációs listájában meghívjuk \texttt{Sikidom} egyetlen, két \texttt{int}-et váró konstruktorát.
\begin{lstlisting}
class Sikidom
{
protected:
	int x, y;
public:
	Sikidom(int x, int y) : x(x), y(y) {}
};

class Kor : public Sikidom
{
	double r;
public:
	Kor(int x, int y, int r) : Sikidom(x,y), r(r) {}
};
\end{lstlisting}
	Ezzel \texttt{Sikidom} konstruktorát meg tudtuk hívni. Mi lesz a copy konstruktorokkal? Hasonló helyzetben, ha egy kört másolunk, a síkidom default copy konstrukora fog meghívódni, kivéve, ha az felül van definiálva. Azaz rendre másolásnál ellőször meghívódik a síkidom copy konstruktora, és utána jön a \texttt{Kor} copy konstruktora. Ebben a sorrendben haladnak konstruktorok is, azonban a destruktorok nem: ott a hívási sorrend pont fordítva van, azaz felfelé halad az öröklődési láncon.
	
	\begin{lstlisting}
class A {};
class B : public A {};
class C : public B {};
class D : public C {};
	\end{lstlisting}
	Konstruktorok hívási sorrendje: \texttt{A -> B -> C -> D}
	
	\medskip
	Előfordulhat olyan helyzet, mikor nem akarjuk, hogy az adott osztályt példányosítsák. Például a fent erre egy módszer volt a tisztán virtuális metódusok használata. Egy másik módszer az, ha az összes konstruktort védetté tesszük.
	\begin{lstlisting}
class Base
{
protected:
	Base() {}
};

class Derived : public Base {};
	\end{lstlisting}
	Ebben az esetben \texttt{Derived} meg tudja hívni \texttt{Base} konstruktorát, így példányosítható lesz, de önmagában \texttt{Base} nem.
	\bigskip
	
	Azonban itt van egy nagy veszélyforrás, hiszen a destruktor is csak akkor hívódik meg jól, ha az virtuális. Amennyiben a destruktorunkat nem tesszük virtuálissá, egy nem definiált viselkedés áldozatai leszünk, azonban közel biztosan az fog bekövetkezi, hogy a bázisosztályok destruktorai nem futnak le.
	
	\smallskip
	Javítsuk is ezt gyorsan:
\begin{lstlisting}
class Sikidom
{
protected:
	int x;
	int y;
public:
	Sikidom(int x, int y) : x(x), y(y) {}
	virtual ~Sikidom() {}
	virtual double terulet() const;
};
\end{lstlisting}
	Bár örökölni közel minden osztályból lehet (ezalól kivétel lehet egy osztályon belüli privátként deklarált inline class-ok, vagy azok, melyek c++11ben \texttt{final}-ként vannak deklarálva), ettől még nem mindig bölcs döntés az. Példaképp örökölni STL konténerekből nagy galibát okozhat, hisz azon destruktora nem virtuális.
	
	\medskip
	Ha biztosra akarunk menni, deklaráljunk minden destruktort virtuálisnak. Azonban tartsuk észben, hogy csakúgy mint minden virtuális függvény, ez futási idejű költséggel párosul. Ha 100\%, hogy az osztályunkból nem fognak örökölni, bátran hagyjuk annak destruktorát \texttt{virtual} kulcsszó nélkül.
	
	\medskip
	A fenti \texttt{A, B, C, D} osztályokat tekintvén, a destruktorok lefutási sorrendje ellentétes:  \texttt{D -> C -> B -> A}
	%TODO Sikidom konstruktorát meg kéne írni rendesen, hogy ne gyak végére jöjjön a pofon hogy az mégse jó.
	
	\bigskip
	
	Forrásként felhasználtam \textsc{Brunner} Tibor honlapján talált információkat: \url{http://bruntib.web.elte.hu/read.php?3,52}.
	
	\subsection{C++11, privát öröklődés, privát adattagok használata, slicing}
	
	\medskip
	C++11-ben lehetőségünk van picit kényelmesebbé tenni az életünket, például az \texttt{override} kulcsszóval, mely garantálja, hogy az adott függvény felülír egy másikat.
	\begin{lstlisting}
class Base
{
public:
	virtual void f();
};

class Derived : public Base
{
public:
	virtual void f() override {}
};
	\end{lstlisting}
	Amennyiben a felülírás mégse sikerült (\texttt{f} nem virtuális, nem egyezik a paraméterlista/konstansság), fordítási idejű hibát kapunk, mely mindig sokkal jobb, mint a futási idejű.
	
	\smallskip
	Hasonlóan hasonló kulcsszó a \texttt{final}, mely megtiltja az öröklődést, így bátran használhatunk nem virtuális destruktorokat.
	
	\begin{lstlisting}
class Base final
{
	public:
	virtual void f();
};

class Derived : public Base {}; //forditasi ideju hiba
	\end{lstlisting}
	Amennyiben örököltünk egy osztályból, mely privát adattagokat tartalmaz, azokhoz nem fogunk tudni hozzáférni. Ilyenkor rá vagyunk kényszerülve get/set függvények alkalmazására, ha azok meg vannak írva (szempont lehet megírásuk szempontjából ez is).
	
	\medskip
	Tartsuk észben, hogy örökölni nem minden második sorban fogunk, így ha alapértelmezett öröklődést is szeretnénk használni, \textbf{mindig írjuk ki azt explicit módon}! Nem csak számunkra, de más számára is jobban érthető lesz így a kód.
	
	\medskip
	A privát öröklődés egy módszere lehet annak, hogy egy tartalmazásos (\textit{has a}) relációt állítsunk fel két osztály között. Privát öröklődés hatására minden metódus és adattag, melyhez hozzáfértünk volna alapból is, továbbra is hozzáférhető marad, azonban a leszármazott osztályból ha öröklünk, később már nem.
	\begin{lstlisting}
class Base
{
public:
	int x;
};

class Derived : private Base
{
public:
	int f() {x = 2;} //ok
};

class Derived2 : private Derived
{
public:
	int f() {x = 2;} //forditasi ideju hiba
};
	\end{lstlisting}
	Továbbá ha az öröklődési láncban privát öröklődés van, akkor már a bázis pointerével nem lehet a leszármazottra hivatkozni (\texttt{Base *b = Derived} fordítási idejű hibát ad).
	\medskip
	
	A slicing-ról már volt szó, azonban veszélyesebb az talán, mint elsőre gondolnánk. Emlékeztetőként, akkor fordul ez elő, ha egy objektumot egy, a bázisosztály objektuménak adjuk \textbf{értékül}. Ilyenkor minden speciális adat elveszik, nem fogunk tudni hivatkozni specifikus metódusokra, de ami még veszélyesebb, hogy a virtuális függvényeket ugyanúgy meg fogjuk tudni hívni, azonban azok mást fognak csinálni, mint amire számítunk.
	\begin{lstlisting}
struct Base
{
	virtual void print() const
	{
		std::cout << "Base!" << std::endl;
	}	
};

struct Derived : public Base
{
	void print() cosnt
	{
		std::cout << "Derived!" << std::endl;
	}	
};

void printRef(const Base& b)
{
	b.print();
}

void printVal(const Base b)
{
	b.print();
}

int main()
{
	Base b;
	Derived d;
	printRef(b); //Base!
	printRef(d); //Derived!
	printVal(b); //Base!
	printVal(d); //Base!
}
	\end{lstlisting}
	Láthatjuk, ebbe a problémába belefutni könnyebb mint gondolnánk. Ez is plusz egy indok, miért preferáljuk a referencia szerinti átvételt, és érték szerint csak akkor veszünk át, ha nagyon indokolt. E téren a kivételezésnél is jobb ha figyelünk!
	\begin{lstlisting}
int main()
{
	try
	{
		throw Derived();
	}
	catch(const Base& b) //referencia!
	{
		b.print(); //Derived!
	}
}
	\end{lstlisting}
	Az öröklődés és a kivételkezelés kapcsolatárólról még sokat lehetne beszélni, de ez egy későbbi lecke témája lesz.
\end{document}
