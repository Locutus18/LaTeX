\documentclass[a4paper,11.5pt]{article}
\usepackage[textwidth=170mm, textheight=230mm, inner=20mm, top=20mm, bottom=30mm]{geometry}
\usepackage[normalem]{ulem}
\usepackage[utf8]{inputenc}
\usepackage[T1]{fontenc}
\PassOptionsToPackage{defaults=hu-min}{magyar.ldf}
\usepackage[magyar]{babel}
\usepackage{amsmath, xcolor, amsthm,amssymb,paralist,array, ellipsis, graphicx}
%\usepackage{marvosym}

\usepackage{listings}
\lstset{
	language=C++, 
	basicstyle=\ttfamily, 
	keywordstyle=\color{blue}\ttfamily, 
	stringstyle=\color{red}\ttfamily,
	tabsize = 4
}

\makeatletter
\renewcommand*{\mathellipsis}{%
	\mathinner{%
		\kern\ellipsisbeforegap%
		{\ldotp}\kern\ellipsisgap%
		{\ldotp}\kern\ellipsisgap%
		{\ldotp}\kern\ellipsisaftergap%
	}%
}
\renewcommand*{\dotsb@}{%
	\mathinner{%
		\kern\ellipsisbeforegap%
		{\cdotp}\kern\ellipsisgap%
		{\cdotp}\kern\ellipsisgap%
		{\cdotp}\kern\ellipsisaftergap%
	}%
}
\renewcommand*{\@cdots}{%
	\mathinner{%
		\kern\ellipsisbeforegap%
		{\cdotp}\kern\ellipsisgap%
		{\cdotp}\kern\ellipsisgap%
		{\cdotp}\kern\ellipsisaftergap%
	}%
}
\renewcommand*{\ellipsis@default}{%
	\ellipsis@before
	\kern\ellipsisbeforegap
	.\kern\ellipsisgap
	.\kern\ellipsisgap
	.\kern\ellipsisgap
	\ellipsis@after\relax}
\renewcommand*{\ellipsis@centered}{%
	\ellipsis@before
	\kern\ellipsisbeforegap
	.\kern\ellipsisgap
	.\kern\ellipsisgap
	.\kern\ellipsisaftergap
	\ellipsis@after\relax}
\AtBeginDocument{%
	\DeclareRobustCommand*{\dots}{%
		\ifmmode\@xp\mdots@\else\@xp\textellipsis\fi}}
\def\ellipsisgap{.1em}
\def\ellipsisbeforegap{.05em}
\def\ellipsisaftergap{.05em}
\makeatother

\usepackage{hyperref}

\begin{document}
	%%%%%%%%%%%RÖVIDÍTÉSEK%%%%%%%%%%
	\setlength\parindent{0pt}
	\def\s{\hspace{0.2mm}\vphantom{\beta}}
	\def\Z{\mathbb{Z}}
	\def\Q{\mathbb{Q}}
	\def\R{\mathbb{R}}
	\def\C{\mathbb{C}}
	\def\N{\mathbb{N}}
	\def\Ra{\overline{\mathbb{R}}}
	
	\def\sume{\displaystyle\sum_{n=1}^{+\infty}}
	\def\sumn{\displaystyle\sum_{n=0}^{+\infty}}
	
	\def\narrow{\underset{n\rightarrow+\infty}{\longrightarrow}}
	\def\limn{\displaystyle\lim_{n\to +\infty}}
	\def\limx{\displaystyle\lim_{x\to +\infty}}
	
	\theoremstyle{definition}
	\newtheorem{theorem}{Tétel}[subsection] 
	
	\theoremstyle{definition}
	\newtheorem{definition}[theorem]{Definíció} 
	\newtheorem{example}[theorem]{Példa} 
	\newtheorem{task}[theorem]{Feladat} 
	\newtheorem{note}[theorem]{Megjegyzés}
	%%%%%%%%%%%%%%%%%%%%%%%%%%%%%%%%%%%%%%%%%%%%%%%%%%%%%%%%%%%%%%%%%%%%%
	\begin{center}
		{\LARGE\textbf{C++}}
		
		{\Large Gyakorlat jegyzet}
		
		1. óra
	\end{center}
	A jegyzetet \textsc{Umann} Kristóf készítette \textsc{Horváth} Gábor gzakorlatán. (\today)
	\section{Bevezető}
	A C++ többek között a hatákonyságáról is híres. Andrei Alexandrescu azt nyilatkozta, hogy amikor a Facebooknál a backend kódján 1\%ot sikerült optimalizálni, több mint 10 évnyi fizetését spórolta meg a cégnek havonta csak az áramköltségen. Nem használ garbage collectort: nincs nem várt szünet a program végrehajtásában a menedzselt nyelvekkel szemben.
	
	\medskip
	Cél: a tárgy során kialakítani a nyelvvel kapcsolatban egy intuíciót, ami segítéségével elkerülhetőek alapvető hibák is. Az előzménytárgyakban az egyszerűség kedvéért gyakran féligazságok hangzottak el, ezeket is helyre kell rakni.
	\subsection{Mi az a C++?}
  Alapvetően a nyelv két összetevőből áll. Az aktuális szabványból és annak implementációiból (fordítók + szabványkönyvtárak). A szabvány, ami meghatározza a nyelv nyelvtanját, valamint azt, hogy mit jelentenek a leforduló programok (többé kevésbé). Emellett a szabvány definiálja a szabványkönyvtárat is, amit mindan szabványos C++ fordító mellé szállítani kell.. Az első C++ szabvány a {C++98} volt. További szabványai: {C++03}, {C++11}, {C++14}, {C++17}.
	
	\medskip
  %% TODO linkek.
  A szabvány alapján számos fordító (implementáció) létezik a C++ kódok fordítására: MSVC (Visual Studio), GCC, Clang. 
	\section{Első óra}
	\subsection{Nem definiált viselkedések}
	\begin{example}\ 
		
		\begin{lstlisting}
int main()
{
	int i = 0;
	std::cout << ++i << ++i << std::endl;
}
		\end{lstlisting}
		
		Lefordítás és futtatás után különböző eredményeket kaphatunk, mert itt az, hogy mikor értékelődik ki a két \texttt{++i} a kifejezésen belül, az \textbf{nem specifikált.} Ha a szabvány nem terjed ki arra, hogy milyen kódot generáljon a fordító, akkor a fordító bármit választhat. 
		
		\medskip
		Gyakran eldönthetetlen előre, hogy mikor mi lesz a legmegfelelőbb megoldás, ezért nem definiál mindent a szabvány.
		
		\medskip
		Ez lehetőséget ad a fordítónak arra, hogy \textbf{optimalizáljon.} 
	\end{example}
	
	A C++ban nem meghatározott, hogy két szekvenciapont között mi az utasítások végrehajtásának a sorrendje.
	
	\medskip
	A C++ban van un. szekvenciapontok, és a szabvány csak azt mondja ki, hogy a szekvenciapont előtti kód hamarabb kerüljön végrehajtásra mint az utánalevő.  Mivel itt az \textit{i} értékadása után és csak az \texttt{std::endl} után van szekvenciapont, így az, hogy milyen sorrendben történjen a kettő közötti kifejezés részkifejezéseinek a kiértékelése, az a fordítóra van bízva.
	
	\medskip
	Az, hogy két részkifejezés szekvenciaponttal történő elválasztás nélkül ugyanazt a memóriaterületet módosítha, \textbf{nem definiált} viselkedést eredményez. Nem definiált viselkedés esetén a fordító vagy a futó program bármit csinálhat. A szabvány semmiféle megkötést nem alkalmaz.
	\begin{note}
		Az a program, amely nem definiált viselkedéseket tartalmaz, hibás.
	\end{note}
	\subsection{Nem specifikált viselkedések}
	Amennyiben a szabvány definiál néhány lehetséges opciót, de a fordítóra bízza, hogy az melyiket választja, akkor \textbf{nem specifikált} viselkedésről beszélünk.
	
	A nem specifikált viselkedés csak akkor probléma, ha a program végeredménye függ tőle. Például a fenti kódot módosíthatjuk a következő képpen:
	
	\begin{example}\ 
		
		\begin{lstlisting}
int main()
{
	int i = 0;
	int j = 0;
	std::cout << ++i << ++j << std::endl;
}
		\end{lstlisting}
	\end{example}
	Bár azt továbbra se tudjuk, hogy \texttt{++i} vagy \texttt{++j} értékelődik ki hamarabb, (\textit{nem specifikált}), de azt biztosan tudjuk, hogy 11-et fog kiírni (a program végeredménye jól \textit{definiált}).
	
	\section{A fordító működése}
	
	\begin{center}
		$\begin{matrix}
    &\fbox{\text{header állomány}}\\
		&\swarrow\searrow&\\
    \fbox{\text{cpp fájl}}&&\fbox{\text{cpp fájl}}\\
		\downarrow&&\downarrow\\
		\fbox{Fordítási egység}&&\fbox{Fordítási egység}\\
		\downarrow&&\downarrow\\
		\fbox{Tárgykód}&&\fbox{Tárgykód}\\
		&\searrow\swarrow&\\
		&\fbox{Végrehajtható állomány}&
	\end{matrix}$
	\end{center}
	\begin{compactitem}
		\item Preprocesszálás
    \item Fordítás (A tárgykód létrehozása)
    \item Linkelés (Szerkesztés)
	\end{compactitem}
	\subsection{Preprocesszor}
	A preprocessor használata a legtöbb esetben kerülendő. Ez alól kivétel a header állományok include-olása. A preprocesszor \textbf{primitív} szabályok alapján dolgozik és \textbf{nyelvfüggetlen.} Mivel semmit nem tud a C++-ról, ezért sokszor a fejlesztő számára meglepő viselkedést okozhat a használata.
	\begin{example}\ 
		alma.h
		\begin{lstlisting}
#define ALMA 5

ALMA ALMA ALMA
		\end{lstlisting}
		Az előfeldolgozott szöveget a \texttt{cpp alma.h} parancs kiadása segítségével tekinthetjük meg.
		
		Az előfeldolgozás eredménye: \texttt{5 5 5}.
	\end{example}
	\begin{example}\
		
		\begin{lstlisting}
#define KORTE

#ifdef KORTE
MEGVAN
#else
KORTE
#endif
		\end{lstlisting}
		Kimenetben csak az van benne, hogy MEGVAN.
	\end{example}
	
	\begin{example}\
		
		\begin{lstlisting}
#define KORTE

#undef KORTE

#ifdef KORTE
MEGVAN
#else
KORTE
#endif
		\end{lstlisting}
		Itt a kimenetben KORTE lesz.
	\end{example}
	Az előfeldolgozót kódrészletek kivágására is lehet hasnzálni. A kimenetben viszont a megfelelő direktívákból visszaállítható az eredeti sorszám.
	\begin{note}\
		\texttt{g++ -save-temps hello.cpp} $\rightarrow$ A fordítás köztes fázísainak a kimentése.
	\end{note}
	Mivel az \texttt{include}-ok szöveg szerinti beillesztést jelentenek, nagyon jelentősen meg tudják növelni a kód méretét, ami a fordítást lassítja. Ezért óvatosan kell vele bánni.
	\begin{example}\ 
		pp.h
		\begin{lstlisting}
#include "pp.h"
		\end{lstlisting}
	\end{example}
		
	Rekurzív inlcude-nál az előfeldolgozó egy bizonyos mélység limit után leállítja a preprocesszálást.
	
  Sok és hosszú include láncok esetén nehéz megakadályozni, hogy kör kerüljön az include gráfba (rekurzív include-ok).
	\begin{note}
		\texttt{\#define KORTE ALMA} jelentése: \texttt{KORTE} cseréje \texttt{ALMA}-ra.
	\end{note}
	\begin{example}\ 
		
		pp.h
		\begin{lstlisting}
#ifndef _PP_H_
#define _PP_H_

FECSKE

#endif
		\end{lstlisting}
		
		alma.h
		\begin{lstlisting}
#include "pp.h"
#include "pp.h"
#include "pp.h"
#include "pp.h"
#include "pp.h"
		\end{lstlisting}
	\end{example}
	
	Egy trükk segítségével megakadályozhatjuk azt, hogy többször be legyen illesztve \texttt{FECSKE}. Itt először megnézzük, hogy \texttt{\_PP\_H\_} szimbólum definiálva van-e. Ha nincs, definiáljuk. Mikor legközelebb ezt meg akarnánk tenni, nem illesztjük be a \texttt{FECSKE}-t, mert \texttt{\#ifndef \_PP\_H\_} kivágja azt a szövegrészt.
	
  Ez az úgy nevezett \textbf{header guard} vagy \textbf{include guard.}
	
	\medskip
	A preprocesszor az itt bemutatottaknál sokkal többet tud, de általában nem érdemes túlhasználni.
	\subsection{Fordítás}
	\begin{example}\
		
		\begin{lstlisting}
int factorial(int n)
{
	if (n <= 0) return 1;
	else return n*factorial(n-1);
}

int main()
{
	std::cout << factorial(5) << std::endl;
}
		\end{lstlisting}
		
		Az optimalizálatlan assembly kódból látható, hogy 5-el lett meghívva a faktoriális függvény.
		\begin{note}
			\texttt{g++ -save-temps hello.cpp -O2}
			
			2-es szintű optimalizálás bekapcsolása.
		\end{note}
		
     Az optimalizált assembly kódból kiolvasható, hogy a kód (kellően friss gcc-vel) a faktoriális kiszámolása helyett a végeredményt (120at) tartalmazza. Így, mivel az eredmény már fordítási időben kiszámolásra került, futási időben nem kell ezzel plusz időt tölteni.
		
		A fordító sokat \textbf{optimalizál.} Az optimalizáció során a szabványos és csak definiált viselkedést tartalmazó kód jelentése nem változhat, viszont sokkal hatékonyabbá válhat.
	\end{example} 
	\begin{note}
		\texttt{-O3} Olyan optimalizálásokat is tartalmazhat, amik agresszívabban kihasználják, ha egy kód nem definiált viselkedéseket tartalmaz.
		
		\texttt{-O2} Kevésbé aggresszív, sokszor a nem szabványos kódot se rontja el.
		
		Mivel nem definiált viselkedésekre rosszul tud reagálni az \texttt{-O3}, így sokszor elég kockázatos használni.
	\end{note}
	
	\subsection{Linkelés}
	\begin{example}\ 
		
		fecske.cpp
		\begin{lstlisting}
void fecske()
{

}
		\end{lstlisting}
		
		main.cpp
		\begin{lstlisting}
int main()
{
	fecske();
	std::cout << factorial(5) << std::endl;
}
		\end{lstlisting}
		
		Ez nem fog lefordulni, mert vagy csak a main.cpp-t, vagy a fecske.cpp-t látja a fordító, egyszerre a két fájlt nem. Megoldás az ha forward deklarálunk, \texttt{void fecske()}-t beillesztjük a main függvény felé. Ekkor a linkelési fázisánál kapunk hibát, mert nem találja a fecske függvény definícióját.
		
    \texttt{g++ -c main.cpp}

		\texttt{g++ -c fecske.cpp}

		Ha linkelés nélkül lefordítom külön a main.cpp-t, a fecske.cpp-t, akkor kapok két object fájlt, és a
		
		\texttt{g++ main.o fecske.o}
		
    linkeli az eredményül kapott tárgykódokat össze. Rövidebb, ha egyből a cpp fájlokat adjuk meg a fordítónak.

		\texttt{g++ main.cpp fecske.cpp}
	\end{example}
	Ha a fecske.cpp-ben sok függvény van, akkor nem célszerű egyesével forward deklarálni őket minden egyes fájlban, ahol használni szeretnénk ezeket a függvényeket. Ennél egyszerűbb egy header fájl megírása, amiben deklaráljuk a fecske.cpp függvényeit.
	
	fecske.h
	\begin{lstlisting}
#ifndef _FECSKE_H_
#define _FECSKE_H_
void fecske();
#endif
	\end{lstlisting}
	Ilyenkor elég a \texttt{fecske.h}-t includeolni.
	
	\begin{definition}
		Függvény definíció: amikor megmondjuk a függvénynek hogy mit csináljon. Egyben deklaráció is.
	\end{definition}
	\begin{definition}
		Függvény deklaráció: a függvény használatáról ad információt. A paraméterek típúsáról, visszatérési értékről és a függvény nevéről.
  \end{definition}
	Szokás a fecske.h-t a fecske.cpp-be is includeolni, mert ha véletlenül ellent mondana egymásnak a definíció a cpp fájlban és a deklaráció a header fájlban akkor a fordító hibát fog jelezni.
	\begin{definition}
		One Definition Rule (ODR): Bármit lehet bármennyiszer deklarálni, azonban ha azok ellentmondanak egymásnak, akkor fordítási hibát kapunk. Definiálni viszont mindent pontosan egyszer kell. A több definíció vagy a definíció hiánya problémát okozhat.
	\end{definition}
	\begin{example}
		
		fecske.h
		\begin{lstlisting}
#ifndef _FECSKE_H_
#define _FECSKE_H_
void fecske();
int macska() {}
#endif
		\end{lstlisting}
		
		Ha több fordítási egységből álló programot fordítunk, akkor a preprocesszor több macska függvény definíciót csinál, és linkeléskor a linker azt látja, hogy egy függvény többször van definiálva, és ez linkelési hibát eredményez.
	\end{example}
	\begin{note}
		A header fájlokba nem szabad definíciókat rakni (bár kivétel létezik, pl. template-ek, inline függvények).
	\end{note}
	\subsection{Implementáció által definiált dolgok}
	A szabvány nem köti meg, hogy egy \textit{int} egy adott platformon mennyi byte-ból álljon. Ez állandó, egy adott platformon egy adott fordító mindig ugyanakkorát hoz létre, de platform/fordítóváltás esetén ez változhat.
	\section{Féligazságok, hazugságok előzménytárgyakból}
	\begin{example}
		A main függvény végrehajtásával kezdődik a program futása?
		\begin{lstlisting}
std::ostream& o = std::cout << "Hello";
int main()
{
	std::cout << "valami";
}
		\end{lstlisting}
		Kimenet: \texttt{Hellovalami}.
		
		Tehát ez nem volt igaz. A program végrehajtásánál az első lépés a globális változók inicializálása. Ennek az oka az, ha \texttt{o}-t akarnám használni a main() függvény első sorában, akkor tudjam (lévén elvárás az, hogy egy globális változó bárhol elérhető legyen). Inicializálatlan változó használata pedig nem definiált viselkedés, ezért fontos már a main végrehajtása előtt inicializálni a globálisokat.
		\begin{lstlisting}
int f()
{
	return 5;
}
int x = f();

int main()
{
	std::cout << "valami";
}
		\end{lstlisting}
    Itt szintén az \texttt{f()} kiértékelése a \texttt{main} függvény meghívása előtt történik.
	\end{example}
	\begin{example}
		A linkelés nem változatathat a program jelentésén?
		
		main.cpp
		\begin{lstlisting}
std::ostream& o = std::cout << "Hello";
int main() {}
		\end{lstlisting}
		fecske.cpp
		\begin{lstlisting}
std::ostream& o = std::cout << " World";
		\end{lstlisting}
		Itt nem specifikált a két globális változók inicializációs sorrendje, és ha más sorredben fordítjuk, mást ír ki.
		
		{\centering \texttt{g++ main.cpp fecske.cpp \quad $\not=$\quad g++ fecske.cpp main.cpp} \par}
	\end{example}
	\begin{note}
		Ez utolsó példa nem jó kódnak számít, mert nem specifikált viselkedést használ ki. A program kimenete nem definiált. Ez is egy jó elrettentő példa, miért nem érdemes globális változókat használni.
	\end{note}
\end{document}
