\documentclass[a4paper,11.5pt]{article}
\usepackage[textwidth=170mm, textheight=230mm, inner=20mm, top=20mm, bottom=30mm]{geometry}
\usepackage[normalem]{ulem}
\usepackage[utf8]{inputenc}
\usepackage[T1]{fontenc}
\PassOptionsToPackage{defaults=hu-min}{magyar.ldf}
\usepackage[magyar]{babel}
\usepackage{amsmath, xcolor, amsthm,amssymb,paralist,array, ellipsis, graphicx}
%\usepackage{marvosym}

\usepackage{listings}
\lstset{
	language=C++, 
	basicstyle=\ttfamily, 
	keywordstyle=\color{blue}\ttfamily, 
	stringstyle=\color{red}\ttfamily,
	tabsize = 4
}

\makeatletter
\renewcommand*{\mathellipsis}{%
	\mathinner{%
		\kern\ellipsisbeforegap%
		{\ldotp}\kern\ellipsisgap%
		{\ldotp}\kern\ellipsisgap%
		{\ldotp}\kern\ellipsisaftergap%
	}%
}
\renewcommand*{\dotsb@}{%
	\mathinner{%
		\kern\ellipsisbeforegap%
		{\cdotp}\kern\ellipsisgap%
		{\cdotp}\kern\ellipsisgap%
		{\cdotp}\kern\ellipsisaftergap%
	}%
}
\renewcommand*{\@cdots}{%
	\mathinner{%
		\kern\ellipsisbeforegap%
		{\cdotp}\kern\ellipsisgap%
		{\cdotp}\kern\ellipsisgap%
		{\cdotp}\kern\ellipsisaftergap%
	}%
}
\renewcommand*{\ellipsis@default}{%
	\ellipsis@before
	\kern\ellipsisbeforegap
	.\kern\ellipsisgap
	.\kern\ellipsisgap
	.\kern\ellipsisgap
	\ellipsis@after\relax}
\renewcommand*{\ellipsis@centered}{%
	\ellipsis@before
	\kern\ellipsisbeforegap
	.\kern\ellipsisgap
	.\kern\ellipsisgap
	.\kern\ellipsisaftergap
	\ellipsis@after\relax}
\AtBeginDocument{%
	\DeclareRobustCommand*{\dots}{%
		\ifmmode\@xp\mdots@\else\@xp\textellipsis\fi}}
\def\ellipsisgap{.1em}
\def\ellipsisbeforegap{.05em}
\def\ellipsisaftergap{.05em}
\makeatother

\usepackage{hyperref}

\begin{document}
	%%%%%%%%%%%RÖVIDÍTÉSEK%%%%%%%%%%
	\setlength\parindent{0pt}
	\def\s{\hspace{0.2mm}\vphantom{\beta}}
	\def\Z{\mathbb{Z}}
	\def\Q{\mathbb{Q}}
	\def\R{\mathbb{R}}
	\def\C{\mathbb{C}}
	\def\N{\mathbb{N}}
	\def\Ra{\overline{\mathbb{R}}}
	
	\def\sume{\displaystyle\sum_{n=1}^{+\infty}}
	\def\sumn{\displaystyle\sum_{n=0}^{+\infty}}
	
	\def\narrow{\underset{n\rightarrow+\infty}{\longrightarrow}}
	\def\limn{\displaystyle\lim_{n\to +\infty}}
	\def\limx{\displaystyle\lim_{x\to +\infty}}
	
	\theoremstyle{definition}
	\newtheorem{theorem}{Tétel}[subsection] 
	
	\theoremstyle{definition}
	\newtheorem{definition}[theorem]{Definíció} 
	\newtheorem{example}[theorem]{Példa} 
	\newtheorem{task}[theorem]{Feladat} 
	\newtheorem{note}[theorem]{Megjegyzés}
	%%%%%%%%%%%%%%%%%%%%%%%%%%%%%%%%%%%%%%%%%%%%%%%%%%%%%%%%%%%%%%%%%%%%%
	\begin{center}
		{\LARGE\textbf{C++}}
		
		{\Large Gyakorlat jegyzet}
		
		7. óra.
	\end{center}
	A jegyzetet \textsc{Umann} Kristóf készítette \textsc{Horváth} Gábor  előadásán. (\today)
	
	\section{Template}
	\subsection{Függvény template-ek}
	Térjünk vissza a régebben megírt swap függvényünkhöz.
	\begin{lstlisting}
void swap(int &a, int &b)
{
	int tmp = a;
	a = b;
	b = tmp;
}
	\end{lstlisting}
	Ez a függvény mind addig jó is, amíg csak \texttt{int}-eket szeretnénk megcserélni, de mi van, ha \texttt{std::string}-eket kéne? A megoldás egyszerű, \textbf{túlterheljük} (\textit{overload}) a \texttt{swap} függvényt.
	\begin{lstlisting}
void swap(std::string &a, std::string &b)
{
	int tmp = a;
	a = b;
	b = tmp;
}
	\end{lstlisting}
	Túlterhelésnek azt nevezzük, amikor két vagy több függvénynek a neve azonos, de a paramétereik különböznek. Tagfüggvényeknél különböző konstansság is számít.
	\begin{lstlisting}
class A
{
public:
	void print() 
	{
		std::cout << "Not const" << std::endl;
	}
	void print() const //overload
	{
		std::cout << "Const" << std::endl;
	}
};

int main()
{
	A a;
	a.print(); //Not const
	
	const A ca;
	ca.print(); //const
}
	\end{lstlisting}
	Azonban gyorsan megállapítható, hogy állandóan egy újabb overloadot létrehozni nem épp ideális megoldás. Ez a kisebb gond, a nagyobb az, hogy a kódismétlés áldozatai leszünk: ha bármi miatt egváltozna a \texttt{swap} belső implementáció, az összes létező swap süggvényben meg kéne ejteni a változtatást. Erre egy megoldás lehet, ha létrehozunk egy sablont, melynek mintájára a fordító maga tud generálni egy megfelelő függvényt.
	\begin{lstlisting}
template <typename T>
void swap(T &a, T &b)
{
	T tmp = a;
	a = b;
	b = tmp;
}
	\end{lstlisting}
	Az így implementált \texttt{swap} függvény egy \textit{template}, és a template paramétere egy típus. Ez alapján a fordító már létre tud hozni hozni megfelelő függvényeket:
	\begin{lstlisting}
int main()
{
	int a = 2, b = 3;
	swap<int>(a, b);
	
	double c = 1.3, d = 7.8;
	swap<double>(c, d);
}
	\end{lstlisting}
	A fordítónak csak annyi dolga van, hogy minden \texttt{T}-t lecseréljen \texttt{int}-re, és már kész is a függvény. A fenti példában mi explicit megmondtuk a fordítónak, hogy \texttt{swap}-ot milyen template paraméterrel {példányosítsa} (\textit{instantiate}), azonban függvényeknél erre nem feltétlenül van szükség: a fordító tudja \texttt{a} és \texttt{b} típusát, így ki tudja találni hogy mit kell behelyettesítenie.
	\begin{lstlisting}
int main()
{
	int a = 2, b = 3;
	swap(a, b);
	
	double c = 1.3, d = 7.8;
	swap(c, d);
}
	\end{lstlisting}
	Ezt a folyamatot (amikor a fordító kitalálja a tempalte paramétert) \textbf{template paraméter dedukciónak} (\textit{template parameter deduction}) hívjuk.
	
	\medskip
	Nem csak típus lehet template paraméter -- bármi ami \textbf{nem} karakterlánc literál vagy lebegőpontos szám.
	\begin{lstlisting}
template <typename T, int ArraySize>
int arraySize(const T (&array)[ArraySize])
{
	return ArraySize;
}

int main()
{
	int i[10];
	std::cout << arraySize(i) << std::endl; //10
}
	\end{lstlisting}
	A fenti kód a 3. gyakorlat végén tett megjegyzésből lehet ismerős. Jól demonstrálja a template paraméter dedukciót.
	\section{Osztály template-ek}
	Nem csak függvények, osztályok is lehentek template-ek melyen nagyon hasonlóan működnek.
	\begin{lstlisting}
#include <iostream>

template <typename T>
class X
{
	void f()
	{
		T t;
		t.foo();
	}
};
class Y
{
	void bar() {}
};
int main()
{
	
}
	\end{lstlisting}
	Ez a kód nagyon úgy tűnhet, hogy nem fog lefordulni, lvén mi soha semmilyen \texttt{foo} tagfüggvényt nem írtunk, de mégis le fog. Ez azért van, mert a template osztályok (és függvények) gyakorlatilag sablonok, amiből mi létrehozhatunk egy konkrét osztályt, és mivel sose példányosítottuk, ez olyan fordítás után, mintha benne se lenne a kódban. Szintaktikus ellenőrzést végez a fordító, pl. zárójelek be vannak-e zárva, pontosvessző nem hiányzik-e stb., de azt, hogy van-e olyan \texttt{T} típus, ami rendelkezik \texttt{foo()} függvénnyel, nem nézi.
\begin{lstlisting}
#include <iostream>

template <typename T>
class X
{
	void f()
	{
		T t;
		t.foo();
	}
};
class Y
{
	void bar() {}
};
int main()
{
	X<Y> x;
}
\end{lstlisting}
	Ekkor már azt várnánk hogy valóban nem fordul le, hisz \texttt{Y}-nak nincs \texttt{foo()} metódusa. Azonban lefordul, mivel az \texttt{f()} függvényt nem hívtuk meg, így nem is példányosult az osztályon belül.
	\begin{lstlisting}
int main()
{
	X<Y> x;
	x.f();
}
	\end{lstlisting}
	Itt már végre kapunk fordítási hibát, mert példányosul \texttt{f()}. Ez jól mutatja, hogy a template-ek lusták, és csak akkor példányosulnak, ha nagyon muszáj.
	\medskip
	
	A template-eknek adhatunk meg alapértelmezett értéket.

\begin{lstlisting}
template <typename T = void> //alapertelmezett parameter
class X
{
	void f()
	{
		T t;
		t.foo();
	}
};
class Y
{
	void bar() {}
};
int main()
{
	X<Y> x;
	X<> x2;
}
\end{lstlisting}
	Ilyenkor nem szükséges megadni template paramétert. Továbbá lehet olyan template osztályunk is, mely egy template-et vár.

\begin{lstlisting}
template <typename T>
class X
{
	void f()
	{
		T t;
		t.foo();
	}
};

template <template <typename> class Templ>
class Z
{
	Templ<int> t;
};

int main()
{
	X<Y> x;
	Z<X> z;
}
\end{lstlisting}
	Ez a \texttt{Templ} egy olyan template, aminek a template paramétere egy típus. Így \texttt{Z}-nek a template paramétere egy olyan template, aminek a template paramétere egy típus. Mivel \texttt{X} egy template, így megadható \texttt{Z}-nek template paraméterként.
	\begin{note}
		Fent a template paraméter listában \texttt{typename} helyett \texttt{class} szerepel. Ezek gyakorlatilag ekvivalensek, mind a kettő azt jelenti, hogy az adott paraméter típus (bár a \texttt{typename} beszédesebb).
	\end{note}
	
	A fenti példákban mindig egy default konstruktort használunk. Helyes lenne-e ez?
\begin{lstlisting}
int main()
{
	X<Y> x();
	X<> y2();
	Z<X> z();
}
\end{lstlisting}
	A kód helyesen lefordul, de nem ugyanaz, mintha nem lenne ott a zárójel. Mivel a c++ nyelvtana nem egyértelmű, más kontextusban ugyanaz a kódrészlet mást jelenthet (egyik legegyszerűbb példa a \texttt{static} kulcsszó), így meg kellett alkotni egy olyan szabályt, miszerint amit deklarációként lehet értelmezni, azt deklarációként \textbf{kell} értelmezni. Itt ezek gyakorlatilag függvénydeklarációk lesznek: Az első esetben például egy olyan függvényt deklarálunk, melynek neve \texttt{x}, \texttt{X<Y>}-al tér vissza és nem vár paramétert. 
	
	Így ha default konstruktort szeretnék meghívni, semmilyen zárójelt nem szabad használni.
	\begin{note}
		C++11ben lehet gömbölyű helyett \{\} zárójelet alkalmazni, melynél ez a probléma nem fordulhat elő. pl: \texttt{X<Y> x\{\};}
	\end{note}
	
	A template-ek paramérének ismertnek kell lennie fordítási időben.
	\begin{lstlisting}
template <int N>
void f() {}

int main()
{
	int n;
	std::cin >> n;
	f<n>(); //hiba, n nem ismert forditasi idoben
}
	\end{lstlisting}
	Ez nyilvánvaló, hisz a template-eknek az a funkciója, hogy a fordító generáljon pélkányokat azok alapján, és a fordítási idő végeztével erre nincs lehetőség.
	
	\medskip
	Fontos még, hogy a template-ek nagyon megnövelik a fordítási időt, így nem mindig éri meg egy olyan függvényt is template-ként megírni, melyet nem feltétlenül muszáj.
	\subsection{Template specializáció}
	Néha szeretnénk, hogy bizonyos speciális behelyettesítéseknél más legyen az implementáció mint az alap sablonban. Ilyenkor szokás \textbf{specializációkat} (\textit{template specialization}) létrehozni:
	
	\begin{lstlisting}
template <class T>
class A
{
	public:
	A()	{ std::cout << "general A" << std::endl; }
};

template <>
class A<int>
{
	public:
	A() { std::cout << "special A" << std::endl; }
};

template <class T>
void f() { std::cout << "general f" << std::endl; }

template<>
void f<int>() { std::cout << "special f" << std::endl; }

int main() 
{
	A<std::string> a1; //general A
	f<std::string>(); //general f
	A<int> a2; //special A
	f<int>(); //special f
}
	\end{lstlisting}
	Mind \texttt{A} osztályhoz, mind \texttt{f} függvényhez látrehoztunk egy speciális esetet, amikor a template paraméter \texttt{int}. Számos okunk lehet arra hogy ezt tegyük: a standark könyvtár megfényesebb példája az \texttt{std::vector} osztály, mely egy template, és van template specializációja \texttt{bool} esetre.
	\begin{note}
		Az \texttt{std::vector<bool>} számos optimalizációkat tartalmazhat (persze nem feltétlenül, hisz ez implementáció függő), általában nem \texttt{bool}-okban tárolja az adatokat, hanem bitekben. Sajnos azonban ez hátrányokkal is jár, például hogy a \texttt{[]} operátor érték és nem referencia szerint ad vissza -- bár valóban hatékonyabb, sok szempontból fejfájást okozhat a használata, így c++17-ben ez a specializáció nem fogja a szabvány részét képezni.
	\end{note}
	Írjunk faktoriális számoló algoritmus template-ek segítségével!
	\begin{lstlisting}
template<int N>                           
struct Fact 
{                             
	static const int val = N*Fact<N-1>::val;
};

template<>                                
struct Fact<0>
{                          
	static const int val = 1;               
};               

int main() 
{                                          
	std::cout << Fact<5>::val << std::endl; //120
}
	\end{lstlisting}
	\texttt{Fact} 4szer példányosul: \texttt{Fact<5>, \ldots, Fact<1>,} majd a legvégén az általunk specializált \texttt{Fact<0>}-t hívja meg.
	
	\smallskip
	Ez fel is hívja a figyelmet a template-ek veszélyeit statikus változók használatakor.
	
	\begin{lstlisting}
template <class T>
class A
{
	static int count;
	public:
	A()
	{
		std::cout << ++count << ' ';
	}
};

template <class T>
int A<T>::count = 0;

int main() 
{
	for(int i = 0; i<5; i++)
	{
		A<int> a;
		A<double> b;
	}
}
	\end{lstlisting}
	Kimenet: \texttt{1 1 2 2 3 3 4 4 5 5}
	
	Bár arra számítanánk, hogy 1-től 10ig lesznek a számok kiírva, ne felejtsük, hogy itt két telejsen különböző osztály fog létrejönni: \texttt{A<int>} és \texttt{A<double>}, így a \texttt{count} adattag hiába osztályszintű, 2 teljesen különböző példánya lesz ennek is: \texttt{A<int>::count} és \texttt{A<double>::count}.
	
	\subsection{Dependent scope}
	
	Lehetőségünk van arra hogy osztályon belül deklaráljunk még egy osztályt. Bár erről bővebben a következő órai jegyzetben lesz szó, egy igen fontos problémát vet fel.
	\begin{lstlisting}
class A
{
public:
	class X {};
};

void f(A a)
{
	A::X x;
}

int main()
{
	A a;
	f(a);
}
	\end{lstlisting}
	Ezzel semmi probléma nincs. Legyen A egy template osztály!
	\begin{lstlisting}
template <class T>
class A
{
	public:
	class X {};
};

template <class T>
void f(A<T> a)
{
	A<T>::X x;
}

int main()
{
	A<int> a;
	f(a);
}
	\end{lstlisting}
	Itt máris bajba jutottunk, a fordító azt a hibát fogja jelezni hogy \texttt{X} egy un. \textbf{dependent scope}-ban van. Ez azt jelenti, hogy attól függően, milyen template paraméterrel példányosítjuk \texttt{A}-t, \texttt{X}-nek lehet más a jelentése. Az alábbi kód ezt jól demonstrálja:
	\begin{lstlisting}
template <typename T>
struct A
{
	class X{};
};

template <>
struct A <int>
{
	static int X;
};

int A<int>::X = 0;

template <typename T>
void f()
{
	A<T>::X;
}
	\end{lstlisting}
	Itt az \texttt{f} függvényben vajon mi lesz \texttt{A<T>::X}? A válasz az hogy nem tudni, hisz ha \texttt{int}-el példányosítunk akkor statikus adattag, ha bármi mással, akkor meg egy típus. Ezért kell a fordítónak biztosítani, hogy a template paramétertől függetlenül garantáltan típust fog oda kerülni. Ezt a \texttt{typename} kulcssszóval tehetjük meg.
	\begin{lstlisting}
template <typename T>
void f()
{
	typename A<T>::X;
}
	\end{lstlisting}
	A \texttt{typename} garantálja a fordítónak, hogy bármi is lesz \texttt{T}, \texttt{A<T>::X} mindenképpen típus lesz. Ha mégis olyan template paramétert adunk meg, aminél ez nem teljesülne (ez esetben \texttt{T = int}) akkor fordítási idejű hibát kapunk.
	\begin{note}
		A fordító általában szokott szólni, hogy a \texttt{typename} kulcsszó hiányzik.
	\end{note}
\end{document}