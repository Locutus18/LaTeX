\documentclass[a4paper,11.5pt]{article}
\usepackage[textwidth=170mm, textheight=230mm, inner=20mm, top=20mm, bottom=30mm]{geometry}
\usepackage[normalem]{ulem}
\usepackage[utf8]{inputenc}
\usepackage[T1]{fontenc}
\PassOptionsToPackage{defaults=hu-min}{magyar.ldf}
\usepackage[magyar]{babel}
\usepackage{amsmath, xcolor, amsthm,amssymb,paralist,array, ellipsis, graphicx, multirow}
%\usepackage{marvosym}

\usepackage{listings}
\lstset{
	language=C++, 
	basicstyle=\ttfamily, 
	keywordstyle=\color{blue}\ttfamily, 
	stringstyle=\color{red}\ttfamily,
	tabsize = 4
}

\makeatletter
\renewcommand*{\mathellipsis}{%
	\mathinner{%
		\kern\ellipsisbeforegap%
		{\ldotp}\kern\ellipsisgap%
		{\ldotp}\kern\ellipsisgap%
		{\ldotp}\kern\ellipsisaftergap%
	}%
}
\renewcommand*{\dotsb@}{%
	\mathinner{%
		\kern\ellipsisbeforegap%
		{\cdotp}\kern\ellipsisgap%
		{\cdotp}\kern\ellipsisgap%
		{\cdotp}\kern\ellipsisaftergap%
	}%
}
\renewcommand*{\@cdots}{%
	\mathinner{%
		\kern\ellipsisbeforegap%
		{\cdotp}\kern\ellipsisgap%
		{\cdotp}\kern\ellipsisgap%
		{\cdotp}\kern\ellipsisaftergap%
	}%
}
\renewcommand*{\ellipsis@default}{%
	\ellipsis@before
	\kern\ellipsisbeforegap
	.\kern\ellipsisgap
	.\kern\ellipsisgap
	.\kern\ellipsisgap
	\ellipsis@after\relax}
\renewcommand*{\ellipsis@centered}{%
	\ellipsis@before
	\kern\ellipsisbeforegap
	.\kern\ellipsisgap
	.\kern\ellipsisgap
	.\kern\ellipsisaftergap
	\ellipsis@after\relax}
\AtBeginDocument{%
	\DeclareRobustCommand*{\dots}{%
		\ifmmode\@xp\mdots@\else\@xp\textellipsis\fi}}
\def\ellipsisgap{.1em}
\def\ellipsisbeforegap{.05em}
\def\ellipsisaftergap{.05em}
\makeatother

\usepackage{hyperref}

\begin{document}
	%%%%%%%%%%%RÖVIDÍTÉSEK%%%%%%%%%%
	\setlength\parindent{0pt}
	\def\s{\hspace{0.2mm}\vphantom{\beta}}
	\def\Z{\mathbb{Z}}
	\def\Q{\mathbb{Q}}
	\def\R{\mathbb{R}}
	\def\C{\mathbb{C}}
	\def\N{\mathbb{N}}
	\def\Ra{\overline{\mathbb{R}}}
	
	\def\sume{\displaystyle\sum_{n=1}^{+\infty}}
	\def\sumn{\displaystyle\sum_{n=0}^{+\infty}}
	
	\def\narrow{\underset{n\rightarrow+\infty}{\longrightarrow}}
	\def\limn{\displaystyle\lim_{n\to +\infty}}
	\def\limx{\displaystyle\lim_{x\to +\infty}}
	
	\theoremstyle{definition}
	\newtheorem{theorem}{Tétel}[subsection] 
	
	\theoremstyle{definition}
	\newtheorem{definition}[theorem]{Definíció} 
	\newtheorem{example}[theorem]{Példa} 
	\newtheorem{task}[theorem]{Feladat} 
	\newtheorem{note}[theorem]{Megjegyzés}
	%TODO ezek így nem lesznek jók.
	%%%%%%%%%%%%%%%%%%%%%%%%%%%%%%%%%%%%%%%%%%%%%%%%%%%%%%%%%%%%%%%%%%%%%
	\begin{center}
		{\LARGE\textbf{C++}}
		
		{\Large Gyakorlat jegyzet}
		
		9. óra.
	\end{center}
	A jegyzetet \textsc{Umann} Kristóf készítette \textsc{Brunner} Tibor és \textsc{Horváth} Gábor előadásán. (\today)
	
	\section{STL konténerek}

	Az \textit{STL} a \textit{Standard Template Library} rövidítése.
	\subsection{Funktorok}
	Mielőtt belevetnénk magunkat az STL-ben lévő algoritmusokba és konténerekbe, fontos megismerkednünk a funktorokkal, melyek a rendezéseknél lesznek majd használatosak.
	\medskip
	
	A funktor egy olyan osztály, melynek túl van terhelve a gömbölyű zárójel operátora (tehát kvázi meg lehet hívni).
	
	\smallskip
	Egy egyszerű példa:
	\begin{lstlisting}
struct S
{
	int x;
	int operator()(int y)
	{
		return x + y;
	}
};
int main()
{
	S s1, s2;
	s1.x = 5;
	s2.x = 8;
	std::cout << s1(2) << std::endl; //7
	std::cout << s2(2) << std::endl; //10
}
	\end{lstlisting}
	Itt gyakorlatilag az objektum függvényként funkcionál. A funktorok azért fontosak, mert megadhatók template paraméterként, és így tudjuk elérni, hogy rendezéshez lehessen őket használni.
	\begin{note}
	Miért használunk funktorokat mondjuk függvénypointerek helyett? Világos, hogyha egyedi rendezést akarunk, akkor muszáj ezt az információt valahogy átadni. A funktorok erre alkalmasabbak, példaképp vegyük ehhez egy olyan template függvényt, melynek egyik template paramétere egy olyan funktort vár, melynek \texttt{()} operátora egy paramétert vár és \texttt{bool}-al tér vissza, és feladata az erre igazat adó elem megkeresése. 
	
	Hogyan tudnánk mi funktorok helyett függvénypointerrel a második páros számot megkeresni vele? Vagy az ötödik 8-al oszthatót? Nos, függvényekkel igencsak nehezen (kb statikus adattagokra kényszerülnénk, vagy egy hasonlóan nem épp elegáns megoldásra), azonban egy funktorban létrehozhatunk egy számlálót, melyet tudunk szépen növelgetni.
	\end{note}
	
	Segíthet a megértésben, ha tekintjük az alábbi példát:
	\begin{lstlisting}
template <class Compr>
bool f(int a, int b, Compr c)
{
	return c(a, b);
}

struct Less
{
	bool operator()(int a, int b) const
	{
		return a < b;
	}
};

int main()
{
	if( f(2,3, Less()) )
		std::cout << "2 kisebb mint 3! ";
	
	int a = 2, b = 2;
	if ( f(a, b, Less()) == false && f(b, a, Less()) == false )
		std::cout << "a es b egyenloek!" << std::endl;
}
	\end{lstlisting}
	Kimenet: \texttt{2 kisebb mint 3! a es b egyenloek!}
	
	Ez a funktor (\texttt{Less}) egy un. \textit{binary predicate}, azaz a gömbölyű zárójel operátora két azonos típusú objektumhoz rendel egy logikai értéket. Figyeljük meg, hogyan adtuk meg a harmadik paramétert: \texttt{Less}-nek meghívtuk a default konstruktorát, de nem adtunk neki változónevet. Itt egy un. névtelen temporális változó jön létre, mely \texttt{f} paramétereinek kiértékelésekor jön létre, majd az \texttt{f} lefutása után meg is semmisül. Figyeljük meg azt is, hogy ebben az esetben érték szerint vettünk át: ez nem csak hatékonyabb, hisz \texttt{Less} mérete minimális (nincs adattagja), de mivel \texttt{Less()} egy jobbérték, csak így, vagy konstans referenciával tudnánk átvenni.
	\smallskip
	
	Ez alapján az egyenlőség is levezethető: \quad $a = b \quad \Leftrightarrow \quad \neg(a<b)\, \wedge\, \neg(b<a)$
	\medskip
	
	Könnyű látni, hogy általánosan beszélve, \texttt{Compr} összehasonlító funktorral $a$ és $b$ \texttt{T} típusú objektumoknál
	\[ a\ \ \text{és}\ \ b\ \ \text{ekvivalens}\quad \Leftrightarrow\quad \neg \,Compr(a, b)\, \wedge\, \neg \,Compr(b, a). \]

	\subsection{Bevezető a konténerekhez}
	C++ban az egyetlen tároló alapértelmezetten a tömb. Azonban az meglehetősen kényelmetlen: a tömb egymás mellett lévő memóriacímek összessége, nem tehetjük meg azt, hogy csak úgy hozzáveszünk 1-1 elemet (mi van ha valaki más már írt oda?). Ezért jobban járunk, ha írunk egy egyedi kontért (pl. az általunk létrehozott \texttt{List}), vagy pedig válogatunk az előre megírt STL konténerek között.
	
	Három konténertípust különböztetünk meg:
	\begin{itemize}
		\item Szekvenciális: Olyan sorrendben tárolja az adattagokat, ahogyan betesszük őket. \\Példa: \texttt{std::vector}, \texttt{std::deque,} \texttt{std::list.}
		\item Asszociatív: Az elemek rendezettek a konténerben.\\ Példa: \texttt{std::set}, \texttt{std::map}, \texttt{std::multiset}, \texttt{std::multimap}.
		\item Konténer adapter: Egy meglévő konténert alakítanak át, általában szogorítanak. Példaképp ha vesszük az \texttt{std::deque} konténert, ami egy kétvégű sor, könnyen tudunk belőle egy egyvégű sort csinálni. Ezen a konténerek általában egy másik konténert tárolnak, és annak a funkcióit szigorítják.
		\\Példa: \texttt{std::queue}, \texttt{std::stack}.
	\end{itemize}
	Minden STL konténer rendelkezik konstans és nem konstans iterátorral, fordított irányú és konstans fordított irányú iterátorral melyre így hivatkozhatunk:
	\begin{center}
		\begin{tabular}{|l|l|l|l|}
			\hline
			Iterátor típus&Ahogy hivatkozhatunk rájuk&Első elem&Past-the-end\\
			\hline
			\hline
			iterátor&\texttt{/*konténer név*/::iterator}&\texttt{begin()}&\texttt{end()}\\
			\hline
			konstans iterátor&\texttt{/*konténer név*/::const\_iterator}&\texttt{cbegin()}&\texttt{cend()}\\
			\hline
			fordított irányú&\multirow{2}{*}{\texttt{/*konténer név*/::reverse\_iterator}}&\texttt{rbegin()}&\texttt{rend()}\\
			iterátor&&&\\
			\hline
			fordított irányú&\multirow{2}{*}{\texttt{/*konténer név*/::reverse\_const\_iterator}}&\texttt{crbegin()}&\texttt{crend()}\\
			konstans iterátor&&&\\
			\hline
		\end{tabular}
	\end{center}
	
	A következő leírásban nem fogunk minden létező tagfüggvénnyel foglalkozni: egyrészt olyan sok van, hogy azt teljesen irreális észben tartani, másrészt sok ilyenhez érdemi hozzáfűznivalót én nehezen tudnék fűzni, így az egyes szekciók végén található link az adott konténerhez.
	
	\begin{note}
		Nagyon fontos képesség az is, hogy valaki hogyan tud utánanézni valaminek, amivel nincs teljesen tisztában, így erősen javallott a \url{cppreference.com}-el való ismerkedés, illetve más helyeken lévő információk böngészése is.
	\end{note}	
	\subsection{vector}
	A \texttt{<vector>} könyvtárban található, maga a konténer az \texttt{std::vector} névre hallgat. Bár a pontos definíciója értelemszerűen implementációfüggő, vannak olyan tulajdonságok, melyeket elvár a szabvány.
	\smallskip
	\begin{lstlisting}
#include <vector>
#inclide <iostream>

int main()
{
	std::vector<int> v;
	v.push_back(5); //v.size() == 1
	v.pop_back(); //v.size() == 0
	
	for (int i = 0; i<10; i++)
		v.push_back(i);
		
	for (std::vector<int>::iterator it = v.begin(); it != v.end(); ++it)
		std::cout << *it << std::endl; // 0 1 2 3 4 5 6 7 8 9
		
	for (int i = 0; i<v.size; i++)
		std::cout << v[i] << std::endl; // 0 1 2 3 4 5 6 7 8 9
}
	\end{lstlisting}
	A \texttt{vector} leggyakrabban dinamikusan lefoglalt tömbben tárolja az adatainkat, viszont a tömb mérete nem növelhető, így ha túl sok elemet akarunk beletenni, a \texttt{vector} lefoglal egy nagyobb memóriaterületet (leggyakrabban kétszer akkorát). Így a \texttt{push\_back}-nek a műveletigénye amortizált konstans: Általában konstans, de ha új memóriaterületet kell lefoglalni és a meglevő elemet átmásolni, akkor lineáris.
	
	\begin{note}
		Számos okból a \texttt{vector} a leggyakoribb választás, ha konténerre van szükségünk. Flexibilitása és gyorsasága kiemelkedő, azonban megvannak a maga gyenge pontjai.
	\end{note}
	
	\smallskip
	Előzménytárgyakból már ismerkedtünk ezzel a konténerrel: az előbb említett \texttt{push\_back} tagfüggvényt szoktuk használni új elem beszúrására a konténer végére, a \texttt{pop\_back} művelettel pedig a konténer végén lévő elemet tujuk kiszedni. 
	
	Feltűnhet, hogy ezek a műveletek, csakúgy mint számos egyéb melyet a cppreference-n olvashatunk, a konténer végére fókuszál, sőt, az elejére vonatkozó műveletek nincsenek is implementálva (nincs semmilyen \texttt{push\_front} vagy \texttt{pop\_front}). Ennek az az oka, hogy az \texttt{std::vector}-nál a konténer végének a módosítása a leghatékonyabb, ha a közepén/elején szeretnénk módosító művelteket végrehajtani, az gyakran a környező elemek odébb másolásával jár.

	\medskip
	Az \texttt{std::vector} tagfüggvényei, mint a nemsoká következő STL algoritmusok, többnyire iterátorokat várnak paraméterül. A következő példában töröljük ki a 4. elemet, majd szúrjuk is vissza!
	\begin{lstlisting}
int main()
{
	std::vector<int> v;
	for (int i = 0; i<10; i++)
		v.push_back(i);
		
	for (int i = 0; i<v.size(); i++)
		std::cout << v[i] << ' '; // 0 1 2 3 4 5 6 7 8 9
		
	std::vector<int>::iterator it = v.begin();
	++it; ++it; ++it;
	v.erase(it);
	for (int i = 0; i<v.size(); i++)
		std::cout << v[i] << ' '; // 0 1 2 4 5 6 7 8 9
	
	it = v.begin();
	++it; ++it; ++it; 
	v.insert(it, 3);
	for (int i = 0; i<v.size(); i++)
		std::cout << v[i] << ' '; // 0 1 2 3 4 5 6 7 8 9
}
	\end{lstlisting}
	\begin{note}
		Nyilván nonszensz, ahogy fent egyesével lépegettünk az iterátorral a negyedik elemig. Erre nemsokára egy értelmesebb megoldást fogunk látni.
	\end{note}
	Az iterátorok kezelésének azonban komoly veszélyei is vannak, a legnagyobb gonosz itt az un. iterátor invalidáció (\textit{iterator invalidation}). Amikor a \texttt{vector} által lefoglalt dinamikus tömb mérete túl kicsi az újabb elemek beszúrásához, és egy újabb tömbbe másolja át őket, minden iterátor, pointer és referencia ami a régebbi tömbre hivatkozott invalidálódik, lévén olyan területre hivatkoznak, melyeket már felszabadultak. 
	\smallskip
	
	Ugyanígy a konténer közepéről történő törlés, vagy a konténer közepére történő beszúrás is iterátor invalidációval jár.
	\smallskip
	
	Az invalidálódott objektumokkal végzett műveletek nem definiált viselkedést eredményeznek.
	\begin{lstlisting}
int main()
{
	std::vector<int> v;
	v.push_back(3);
	std::vector<int>::iterator it = v.begin();
	
	for (int i = 0; i<1000; i++)
		v.push_back(i);
		
	*it = 10; // nem definialt viselkedes
}
	\end{lstlisting}
	Az invalidáció elkerülése végett számos tagfüggvény egy iterátorral tér vissza, erre lehet példa az \texttt{insert}, mely az új elemre, vagy az \texttt{erase}, mely az utolsó eltávolított elem rákövetkezőjére hivatkozik.
	
	\smallskip
	Töröljünk egy vektorból minden páratlan számot!
	\begin{lstlisting}
int main()
{
	std::set<int> c = {1, 2, 3, 4, 5, 6, 7, 8, 9};
	for(std::vector<int>::iterator it = c.begin(); it != c.end(); )
	{
		if(*it % 2 == 1)
			it = c.erase(it);
		else
			++it;
	}
}
	\end{lstlisting}
	
	\medskip
	Link: \url{http://en.cppreference.com/w/cpp/container/vector}.
	
	\begin{note}
		Az \texttt{std::deque} működése ez alapján triviális, így annak megismerését az olvasóra bízom.
	\end{note}
	\subsection{set}
	
	Az \texttt{std::set} a \texttt{<set>} könyvtárban található. E konténernek nincs \texttt{push\_back} tagfüggvénye, helyette az \texttt{insert} függvény használatos elem betevésére. 
	\medskip
	
	Ez a konténer a matematikai halmazra hasonlít: egyedi elemet tárol, és ha egy eleve bent lévő elemmel ekvivalenset próbálnánk beszúrni, nem történne semmi. Ezen kívül a szabvány azt is megköveteli, hogy ez a konténer rendezett legyen, így a tárolandó \texttt{T} típus mellett egy olyan funktort is vár template paramterül, melynek \texttt{()} operátora két \texttt{T} típust vár paraméterül és logikai értékkel tér vissza.
	\smallskip
	
	Ez utóbbi template paramétert nem kell feltétlenül megadni, alapértelmezetten ugyanis az \texttt{std::set} az \texttt{std::less<T>} alapján rendez, mely gyakorlatilag az \texttt{<} operátorral ekvivalens.
	\smallskip
	
	Az \texttt{std::set} általában piros-fekete bináris faként van implementálva, a hatékonyság végett.
	\begin{note}
		A szabvány elvárja, hogy az \texttt{insert} műveletigénye logaritmikus legyen.
	\end{note}
	\begin{lstlisting}
#include <set>
#include <iostream>

struct Circle
{
	int x, y, r;
	Circle(int _x, int _y, int _r) : x(_x), y(_y), r(_r) {}
};

std::ostream& operator<<(std::ostream& os, const Circle &c)
{
	os << c.x << ' ' << c.y << ' ' << c.r;
	return os;
}

bool operator<(const Circle &lhs, const Circle &rhs)
{
	return lhs.r < rhs.r;
}

struct LessByX
{
	bool operator()(const Circle &lhs, const Circle &rhs) const
	{
		return lhs.x < rhs.x;
	}
};

int main()
{
	std::set<int> si;
	for(int i = 10; i>0; i--) //forditva!
		si.insert(i);
	//torlesrol majd kesobb
	//si.size() == 10
	si.insert(5); //si.size() == 10
	
	for(std::set<int>::iterator it = si.begin(); it != si.end(); ++it)
		std::cout << *it << ' '; // 0 1 2 3 4 5 6 7 8 9
	
	std::set<Circle> sk;
	sk.insert(Circle(3, 2, 2));
	sk.insert(Circle(1, 2, 1));
	sk.insert(Circle(2, 2, 3));
	for(std::set<Circle>::iterator it = sk.begin(); it != sk.end(); ++it)
		std::cout << *it << ", "; // 1 2 1, 3 2 2, 2 2 3
	
	std::set<Circle, LessByX> skr;
	skr.insert(Circle(3, 2, 2));
	skr.insert(Circle(1, 2, 1));
	skr.insert(Circle(2, 2, 3));
	for(std::set<Circle, LessByX>::iterator it = skr.begin(); 
												   it != skr.end(); ++it)
		std::cout << *it << ", "; // 1 2 1, 2 2 3, 3 2 2
}	
	\end{lstlisting}

	Egy egyedi rendezés használata veszélyekkel is járhat azonban.
	\begin{lstlisting}
struct strlen
{
	bool operator()(const std::string &lhs, const std::string &rhs) const
	{
		return lhs.length() < rhs.length();
	}
};

int main()
{
	std::set<std::string, strlen> s;
	s.insert("C++");
	s.insert("Java");
	s.insert("Haskell");
	std::cout << s.count("ADA") << std::endl; // 1
}
	\end{lstlisting}
	A \texttt{count} tagfüggvény megszámolja, hány \texttt{ADA}-val \textbf{ekvivalens} elem van a \texttt{set}ben. Mivel a funktorunk úgy hasonlít össze két elemet, hogy megnézi, melyik string hossza nagyobb, \texttt{C++} és az \texttt{ADA} ekvivalens lesz, így legnagyobb meglepődésünkre 1-et fog ez a program kiírni!
	%miért pocakos a dékán úr? mert has kell
\begin{lstlisting}
int main()
{
	std::set<std::string, strlen> s;
	s.insert("C++");
	s.insert("Java");
	s.insert("Haskell");
	s.insert("GOD");
	std::cout << s.size() << std::endl; // 3
}
\end{lstlisting}
	Most értelemszerűen azt várnánk, hogy a méret 4 legyen, de mégis 3at ír ki, mert az \texttt{insert} függvény nem teszi be az \texttt{GOD}-ot, lévén az a rendezés szerint az a \texttt{C++}al ekvivalens.
\begin{lstlisting}
struct strlenWrong
{
	bool operator()(const std::string &lhs, const std::string &rhs) const
	{
		return lhs.length() <= rhs.length(); //ekvivalensek is lehetnek
	}
};

int main()
{
	std::set<std::string, strlenWrong> s;
	s.insert("C++");
	s.insert("Java");
	s.insert("Haskell");
	s.insert("ADA");
	std::cout << s.count("ADA") << std::endl; // 0
	std::cout << s.size() << std::endl; // 4
}
\end{lstlisting}
	Most figyeljük meg (emlékezzünk vissza dimatra), hogy egy szigorú rendezés helyett egy gyenge rendezést definiáltunk! Miért lehet ez problémás?
	
	\smallskip
	Helyettesítsünk be a rendezésünkbe.
	\[ \neg\big(\,a.length() \leq b.length()\,\big)\quad \wedge \quad \neg\big(\,b.length() \leq a.length()\,\big) \]
	Itt ha Ha a \texttt{C++}t és az \texttt{ADA}t behelyettesítsük, látjuk, hogy ez a formula hamisat ad, így nem bizonyulnak majd ekvivalensnek.
	
	Ha nem szigorú részben rendezést használunk, hanem gyengét, akkor a reflexivitást is elvesztjük. Ez azt jelenti, hogy egy elem nem lehet ekvivalens önmagával!
	
	Könnyen látható hogy a rendezésünk nem épp jó, mivel ha ekvivalenciát vizsgálunk, sose fog igazat adni, sose tudjuk meg, egy adott benne van-e, az elemek törlése se teljesen jó így.
	
	Link: \url{http://en.cppreference.com/w/cpp/container/set}
	
	\begin{note}
		Az \texttt{std::multiset} működése az eddigiek alapján triviális, így annak megismerését ismét az olvasóra bízom.
	\end{note}
	\subsection{list}
	Ez a konténer nagyon hasonló ahhoz, mint amit mi írtunk, csak lehet üres, és kétirányú.
	\smallskip
	
	Feltűnő lehet, hogy az \texttt{std::vector}-ral szemben ennek a konténernek nincs \texttt{[]} operátora, hiszen mire egy adott elemet elérünk, szépen el kell oda lépegetni egyesével. A szögletes zárójel operátort csak akkor szokás írni, hogy ha nagyon hatékony, lehetőleg konstans műveletigényű az adott elem lekérdezése, de ez a listánál nem teljesül.
	\begin{note}
		Bár elméletben gyorsabb egy láncolt listába beszúrni, de ez a gyakorlatban általában mégsem igaz. Azért, mert az egy dolog, hogy az elemeket egyesével odébb kell tolni egy vektorban, azonban mivel abban szekvenciálisan vannak a memóriacímek egymás mellett, és a processzor nem egyesével adja vissza az elemeit, hanem egy nagyobb blokkot mutat belőle, a vector sokkal gyorsabban végrehajtja ezeket a műveleteket, míg a listánál muszáj egyik címről a másikra haladni, ami sokkal költségesebb tud lenni.
	\end{note}
	\begin{note}
		Amennyiben sok elemből álló listánk van, és azok nagy objektumokat tárolnak, hatékonyabb lehet a \texttt{vector}-nál.
	\end{note}
	Hogyan tudunk akkor a harmadik elemre ugrani?
	\begin{lstlisting}
#include <iostream>
#include <list>

int main()
{
	std::list<int> l;
	for (int i = 0; i < 5; i++)
	l.push_back(i);
	std::list<int>::iterator it = l.begin();
	++it; ++it;
	std::cout << *it << std::endl; // 2
}
	\end{lstlisting}
	Ennél kell hogy legyen kényelmesebb módszer is. Később, amikor az STL algoritmusokkal foglalkozunk, fogunk látni erre példát.
	%TODO std::advance
	
	\medskip
	Ha böngészünk cppreference-en, feltűnhet, hogy az \texttt{std::vector}-ral szemben azonban létezik \texttt{push\_front} és \texttt{pop\_front} metódus: ezeket az \texttt{std::vector}-ral ellentétben konstans idő alatt el lehet végezni (elegendő a fejelemet a rákövetkező elemre állítani és kész is).
	\medskip
	
	Link: \url{http://en.cppreference.com/w/cpp/container/list}
	\subsection{map}
	Az \texttt{std::map} a \texttt{<map>} könyvtárban található. Ez egy egyedi kulcs-érték párokat tároló konténer, mely minden kulcshoz egy értéket rendel. Működése nagyon hasonlít az \texttt{std::set}-ére, annyi különbséggel, hogy párokat tárol (így két kötelező template paramétere van), és mindig a kulcs szerint rendez.
	\smallskip
	
	Van egy speciális beszúró függvénye is:
\begin{lstlisting}
#include <map>
#include <iostream>

int main()
{
	std::map<std::string, int> m;
	m["Hello"] = 42;
	m["xyz"] = 8;
	m["Hello"] = 9;
	std::cout << m.size() << std::endl; // 2
	std::cout << m["Hello"] << std::endl; // 9
}
\end{lstlisting}
	A \texttt{[]} operátor egy új kulcsot hoz létre, és ahhoz rendel egy új értéket. Az utolsó sornál mivel a \texttt{''Hello''}-t már tartalmazza a map, így annak az értékét felülírja.
	
	\medskip
	Sajnos ez az operátor azonban okozhat pár kellemetlen meglepetést is.
	\begin{lstlisting}
int main()
{
	std::map<std::string, int> m;
	std::cout << m.size() << std::endl; // 0
	if(m["c++"] != 0)
		std::cout << "nem 0" << std::endl;
	std::cout << m.size() << std::endl; // 1
}
	\end{lstlisting}
	A \texttt{[]} operátor úgy működik, hogy ha a \texttt{map} nem tartalmazza a paraméterként kapott kulcsot, akkor létrehozza és beszúrja, ha benne van, visszaadja az értékét. Bár mi nem tettük bele azt hogy \texttt{c++} szándékosan, de pusztán azzal, hogy rákérdeztünk, akaratlanul is megtettük. (Figyeljük meg, hogy akkor tudnánk új elemet hozzátenni, ha azt írnánk pl. hogy \texttt{m["C++"] = 3}; azonban mi nem adtunk meg ehhez a kulcshoz értéket. Ilyenkor a map alapértelmezett értéket rendel a kulcshoz, szám típusoknál 0-t, így be fog szúrni egy \texttt{("c++", 0)} párt.)
	\begin{lstlisting}
bool contains(const std::map<std::string, int> &m)
{
	//m["c++"]; forditasi hiba: [] operator nem konstans fuggveny.
	return m.find("C++") != m.end();
}
	\end{lstlisting}
	Ez a függvény helyesen vizsgálja, hogy a \texttt{c++} benne van-e. (Ha cppreference-en rákeresünk, a \texttt{map}-nek a \texttt{find} nevű tagfüggvénye egy iterátort ad vissza a talált elemre, illetve visszaadja egy past-the-end iterátort ha nem találja meg.)
	\medskip
	
	\texttt{std::map}-ban az iterálás kicsit trükkösebb, ugyanis \texttt{std::map} párokat tárol, méghozzá egy más alapból megírt struktúrát használ, az \texttt{std::pair}-t, ami kb. így néz ki:
	\begin{lstlisting}
template <typename T1, typename T2>
struct pair
{
	T1 first;
	T2 second;
};
	\end{lstlisting}
	Így az az adott kulcs ill. érték lekérdezése így fog kinézni:
	\begin{lstlisting}
for(std::map<std::string, int>::iterator i = m.begin(); i != m.end(); i++)
{
	std::cout << i->first << " " << i->second;
}
	\end{lstlisting}
		
\end{document}
