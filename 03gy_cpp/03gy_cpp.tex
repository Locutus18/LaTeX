\documentclass[a4paper,11.5pt]{article}
\usepackage[textwidth=170mm, textheight=230mm, inner=20mm, top=20mm, bottom=30mm]{geometry}
\usepackage[normalem]{ulem}
\usepackage[utf8]{inputenc}
\usepackage[T1]{fontenc}
\PassOptionsToPackage{defaults=hu-min}{magyar.ldf}
\usepackage[magyar]{babel}
\usepackage{amsmath, xcolor, amsthm,amssymb,paralist,array, ellipsis, graphicx}
%\usepackage{marvosym}

\usepackage{listings}
\lstset{
	language=C++, 
	basicstyle=\ttfamily, 
	keywordstyle=\color{blue}\ttfamily, 
	stringstyle=\color{red}\ttfamily,
	tabsize = 4
}

\makeatletter
\renewcommand*{\mathellipsis}{%
	\mathinner{%
		\kern\ellipsisbeforegap%
		{\ldotp}\kern\ellipsisgap%
		{\ldotp}\kern\ellipsisgap%
		{\ldotp}\kern\ellipsisaftergap%
	}%
}
\renewcommand*{\dotsb@}{%
	\mathinner{%
		\kern\ellipsisbeforegap%
		{\cdotp}\kern\ellipsisgap%
		{\cdotp}\kern\ellipsisgap%
		{\cdotp}\kern\ellipsisaftergap%
	}%
}
\renewcommand*{\@cdots}{%
	\mathinner{%
		\kern\ellipsisbeforegap%
		{\cdotp}\kern\ellipsisgap%
		{\cdotp}\kern\ellipsisgap%
		{\cdotp}\kern\ellipsisaftergap%
	}%
}
\renewcommand*{\ellipsis@default}{%
	\ellipsis@before
	\kern\ellipsisbeforegap
	.\kern\ellipsisgap
	.\kern\ellipsisgap
	.\kern\ellipsisgap
	\ellipsis@after\relax}
\renewcommand*{\ellipsis@centered}{%
	\ellipsis@before
	\kern\ellipsisbeforegap
	.\kern\ellipsisgap
	.\kern\ellipsisgap
	.\kern\ellipsisaftergap
	\ellipsis@after\relax}
\AtBeginDocument{%
	\DeclareRobustCommand*{\dots}{%
		\ifmmode\@xp\mdots@\else\@xp\textellipsis\fi}}
\def\ellipsisgap{.1em}
\def\ellipsisbeforegap{.05em}
\def\ellipsisaftergap{.05em}
\makeatother

\usepackage{hyperref}

\begin{document}
	%%%%%%%%%%%RÖVIDÍTÉSEK%%%%%%%%%%
	\setlength\parindent{0pt}
	\def\s{\hspace{0.2mm}\vphantom{\beta}}
	\def\Z{\mathbb{Z}}
	\def\Q{\mathbb{Q}}
	\def\R{\mathbb{R}}
	\def\C{\mathbb{C}}
	\def\N{\mathbb{N}}
	\def\Ra{\overline{\mathbb{R}}}
	
	\def\sume{\displaystyle\sum_{n=1}^{+\infty}}
	\def\sumn{\displaystyle\sum_{n=0}^{+\infty}}
	
	\def\narrow{\underset{n\rightarrow+\infty}{\longrightarrow}}
	\def\limn{\displaystyle\lim_{n\to +\infty}}
	\def\limx{\displaystyle\lim_{x\to +\infty}}
	
	\theoremstyle{definition}
	\newtheorem{theorem}{Tétel}[subsection] 
	
	\theoremstyle{definition}
	\newtheorem{definition}[theorem]{Definíció} 
	\newtheorem{example}[theorem]{Példa} 
	\newtheorem{task}[theorem]{Feladat} 
	\newtheorem{note}[theorem]{Megjegyzés}
	%%%%%%%%%%%%%%%%%%%%%%%%%%%%%%%%%%%%%%%%%%%%%%%%%%%%%%%%%%%%%%%%%%%%%
	\begin{center}
		{\LARGE\textbf{C++}}
		
		{\Large Gyakorlat jegyzet}
		
		3 óra.
	\end{center}
	A jegyzetet \textsc{Umann} Kristóf készítette \textsc{Horváth} Gábor  előadásán. (\today)
	
	\section{Érték szerinti paraméter átvétel}
	Próbáljuk megvalósítani a swap függvényt!
		\begin{lstlisting}
#include <iostream>
void swap(int a, int b)
{
	int tmp = a;
	a = b;
	b = a;
}

int main()
{
	int c = 5, d = 8;
	swap(c, d);
	std::cout << c << ' ' << d << std::endl;
}
		\end{lstlisting}		
		A program kimenete \texttt{5 8} ment. Ez egy teljesen jól definiált viselkedés. Ez azért van, mert itt \textbf{érték} szerint vettük át (\textit{pass by value}) \texttt{a} és \texttt{b} változót. A következő ábrán, mely egy stack-et ábrázol (amiben a c++ tárolja ezeket a változókat), megfigyelhetjük miért is nem. Képzeljük el, ahogy ebbe a verembe a kódunk elrakja a \texttt{c} és \texttt{d} változókat. Majd meghívja a \texttt{swap} függvényt, melyben létrehozott \texttt{a} és \texttt{b} változókat ismét behelyezi. Bár a függvényre lokális \texttt{a} és \texttt{b} változókat megcseréli, de a függvényhívás után ezeket ki is törli a stackből.
		\begin{center}
			\begin{tabular}{r|c|l}
				&&\\
				&&\\
				\cline{2-2}
				b&~~~~~8~~~~~~&\\
				\cline{2-2}
				a&5&\\
				\cline{2-2}
				\hline
				\hline
				d&8& Innentől jön a main fv.\\
				\cline{2-2}
				c&5&\\
				\cline{2-2}
			\end{tabular}
		\end{center}
		C++ban alapértelmezett a paraméterátadás függvényeknél érték szerint történik.
		
		\begin{lstlisting}
//...
int main()
{
	int c = 5, d = 8;
	//...
	int *p = &c;
}
		\end{lstlisting}
		Itt \texttt{p} egy mutató (\textit{pointer}), mely egy \texttt{int} típusra mutat. Ahhoz, hogy értéket tudjunk adni egy mutatónak, egy memóriacímet (\textit{referenciát}) kell neki értékül adni, amire rá tud mutatni, erre való a \textbf{referáló operátor}, a \&. Ha a mutató által \textit{mutatott értéket} szeretnénk módosítani, akkor dereferálnunk kell a \textbf{dereferáló operátor}ral, ami a *.
		\begin{lstlisting}
int *p = &c; //referaljuk c-t
*p = 4; //dereferaljuk p-t
p = &d;
*p = 7;
		\end{lstlisting}
		Rendre: pointer inicializálása, pointer által mutatott érték módosítása, pointer átállítása másik memóriacíme, és a mutatott érték módosítása.
		\begin{lstlisting}
void swap(int *a, int *b)
{
	int tmp = *a;
	*a = *b;
	*b = tmp;
}
		\end{lstlisting}
		
		Amennyiben ezt a függvényt hívjuk meg, minden oké lesz. De ehhez fontos, hogy ne simán \texttt{swap(c, d)}-t írjunk, az ugyanis nem fog működni, mert a \texttt{c} és \texttt{d} típusa \texttt{int}, és nem \texttt{int*}. Ahhoz, hogy értéket adjunk egy pointernek, a \texttt{c}-hez és \texttt{d}-hez tartozó memóriacímeket kell átadni, így \texttt{swap(\&c, \&d)} lesz a helyes.
		\bigskip
		
		Mivel a függvény most konkrétan a \texttt{c} és \texttt{d} változókkal dolgozik, így valóban meg fogja tudni cserélni az értéküket. Azonban ez még mindig \textbf{érték szerinti} átadás, mert most nem konkrét értéket, hanem a memóriacímet másoltuk át.
		
		\begin{lstlisting}
void swapWrong2(int *a, int *b)
{
	int *tmp = a;
	a = b;
	b = tmp;
}
		\end{lstlisting}
		Ebben a példában nem a pointerek által mutatott értéket, hanem magukat a pointereket cseréljük meg. Itt annyi fog csupán történni, hogy a függvény beljesében \texttt{a} és \texttt{b} pointer másra fog mutatni. De annak értéke nem változik.
		
		\medskip
		\texttt{swap(\&a, 6)} nem fordul le, mert 6 konstans, ráadásul \texttt{int}, és nem egy memóriacím.
	\section{Pointer aritmetika.}
		\begin{lstlisting}
const int ci = 6;
int *p = &ci;
		\end{lstlisting}
		Ez nem fordul le, mert \texttt{ci} konstans, de \texttt{p} nem egy nem konstans pointert. Ez sértené a c++ban ismert \textbf{konstans korrektséget} (const correctness). Itt a probléma az lenne, hogy ha rá tudnénk mutatni, akkor hiába lenne \texttt{ci} konstans, tudnánk módosítani \texttt{p}-n keresztül.
		\begin{lstlisting}
const int ci = 6;
const int *p = &ci;
std::cout << *p << std::endl;
		\end{lstlisting}
		Ez már jó lesz, mert a \texttt{p} egy konstansra mutató pointer, azaz tud mutatni olyan változókra, melyek konstansok. Egy konstansra mutató pointer \textbf{nem tudja megváltoztatni} a mutatott memóriacím értékét. Viszont egy konstansra mutató pointer még tud más memóriacímekre mutatni.
		\begin{lstlisting}
const int ci = 6;
const int *p = &ci;
std::cout << *p << std::endl;

int c = 5;
p = &c;
		\end{lstlisting}
		Ez teljesen szabályos, konstansra mutató pointerrel nem konstans értékre mutatunk. Viszont figyelem, \texttt{c} nem konstans, azt továbbra is tudjuk módosítani (Csak nem \texttt{p}-n keresztül)! Ez meglepetést okozhat, hogyha egy konstans pointer kezelése közben (mely által mutatott terület értékétől nem várnánk hogy változzon) a mutatott cím értéke megváltozik.
		\begin{lstlisting}
const int *p = &ci;
int c = 5;
p = &c;
c = 5;
		\end{lstlisting}
		Szintaktikailag a *-ot sok helyre írhatjuk.
		\begin{lstlisting}
const int *p;
int const *p;
		\end{lstlisting}
		Egy kettő ugyanaz, mint fentebb láthattuk.
		\begin{lstlisting}
int * const p;
const int * const p;
int const * const p;
		\end{lstlisting} 
		Amennyiben a * után van a \texttt{const}, akkor egy \textbf{konstans pointert kapunk}, mely megváltoztathatja a mutatott értéket, de nem mutathat másra. (konstans pointer \quad $\not=$\quad konstanra mutató pointer)
		Mutatóra mutató mutatók is léteznek.
		\begin{lstlisting}
const int **pp = &p;
const int *** const cppp = &pp;
int *const ** const dsfdsf = NULL;
		\end{lstlisting}
		Legalul egy integerre mutató const mutatóra mutató mutatóra mutató konstans mutatót látunk (ebben nem vagyok biztos hogy jól írtam le, gondolom nem meglepő miért).
	\begin{note}
		Mutatóra mutató mutató (**) még előfordul, de komplikáltabb ritkán.
	\end{note}
	\section{Tömbök}
		
		\begin{lstlisting}
#include <iostream>
int main()
{
	int i = 5;
	int t[] = {5,4,3,2,1};
}
		\end{lstlisting}
		Ez egy 5 elemű \textbf{tömb}. Nézzük meg, mekkora a mérete (figyelem, ez \textbf{implementációfüggő!})!
		\begin{lstlisting}
std::cout << sizeof(i) << std::endl;
std::cout << sizeof(t) << std::endl;
		\end{lstlisting}
		Azt látjuk, hogy mindig ötszöröse lesz a \texttt{t} az \texttt{i}-nek. Azaz a tömbök tiszta adatok.  Stacken ábrázolja így képzeljük el:
		\begin{center}
			\begin{tabular}{r|c|l}
				&&\\
				\cline{2-2}
				t[4]&&\\
				\cline{2-2}
				t[3]&~~~~~~~~~~~&\\
				\cline{2-2}
				t[2]&&\\
				\cline{2-2}
				t[1]&&\\
				\cline{2-2}
				t[0]&&\\
				\cline{2-2}
			\end{tabular}
		\end{center}
		Irassuk ki a a tömb elemeit! (de ezt basszuk is el!)
		\begin{lstlisting}
for (int i = 0; i < 6; i++) //nem 6 elemes
{
	std::cout << t[i] << std::endl;
}
		\end{lstlisting} 
		Itt előre látható, hogy túl fogunk indexelni. Ez így egy {nem definiált viselkedés}hez vezet. Várhatóan valamilyen random memóriaszemetet fog kiolvasni (vagy olvashat ki, lévén nem definiált), és sose tudhatjuk pontosan mit.
		
		Most növeljük meg az elemeket, és menjünk el egészen 100ig!
		\begin{lstlisting}
for (int i = 0; i < 100; i++)
{
	++t[i];
}
std::cout << "sajt" << std::endl;
		\end{lstlisting} 
		Ez szintén nem definiált. Mivel olyan memóriaterületeket szeretnénk módosítani, melyeket nem foglaltunk le a programunknak, bajba juthatunk. Itt az órán a {sajt} szöveg ki lehet írva, mégis kaptunk szegmentálási hiba (\textit{segmentation fault}) hibaüzenetet az oprendszertől.
		
		\begin{lstlisting}
for (int i = 0; i < 100000; i++)
{
	++t[i];
}
std::cout << "sajt" << std::endl;
		\end{lstlisting} 
		Itt már (legalábbis ebben az esetben) előbb vágta magát hanyat a program, mielőtt sajt-ot ki tudta volna íratni. Ez jól demonstrálja, hogy ugyanazt  a hibát követtük el, de más volt a végeredmény. Ezért igazán veszélyesek a nem definiált viselkedések.
		\begin{lstlisting}
#include <iostream>
#include <string>

int main()
{
	int t[] = {5,4,3,2,1};
	int isAdmin = 0;
	std:string name;
	std::cin >> name;
	for (int i = 0; i < name.size(); ++i)
	{
		t[i] = 1;
	}
	if (name == "pityu")
		isAdmin = 1;
	std::cout << "Admin?: " << (isAdmin != 0 ) << std::endl;
}
		\end{lstlisting}
		Ha a programnak \texttt{pityu}-t adunk meg amikor be akarja olvasni \texttt{name}-et, akkor minden a legnagyobb rendben. De mivel a forráskódot ismerjük, azért hogyha nagyon hosszú nevet adnánk (nagyobb mint 5), akkor a túlindexelés miatt ki tudjuk használni a nem definiált viselkedéseket, és az is előfordulhat, hogy az \texttt{isAdmin} memóriacímére írunk, és elérjük hogy akkor is adminnak higgyen minket, ha nem vagyunk azok.
		\medskip
		
		Hogyan lehet ezeket a hibákat elkerülni? Túl azon, hogy nagyon figyelni kell, vannak programok amik segítenek nekünk. Ehhez használhatunk \texttt{sanitizer}-eket. Ezek picit módosítanak a kódunkon, és amennyiben futási időben bizonyos nem definiált viselkedéseket követne el, pl. itt a túlindexelés, leütné a programunkat. Használatukhoz elég egy extra paranccsal fordítanunk:
		
		{\centering\texttt{g++ main.cpp -fsanitize=address}\par }
		
		De sajnos ez is csak akkor tud segíteni, ha a probléma előfordul (azaz futási időben, nem fordítási időben ellenőriz). Amennyiben előfordul viszont, elég pontos leírást tudunk kapni arról, hogy merre van a probléma.
		
		{\centering \texttt{g++ main.cpp -Wall -Wextra} \par}
		
		Ez a 2 parancs szintén extra ellenőrzéseket vezet be, de nem változtatják meg a kódot, csak fordítási időben ellenőriznek.
	\subsection{Függvény pointerek}
	C++ban lehetőségünk van arra is, hogy függvényeket adjunk át paraméternek.
		\begin{lstlisting}
int add(int a, int b)
{
	return a + b;
}

int mul(int a, int b)
{
	return a * b;
}

int reduce(int *start, int size, int initial, int (*op)(int, int))
{
	int ret = initial;
	for (int i = 0; i < size; i++)
	{
		ret = (*op)(ret, start[i]);
	}
	return ret;
}

int main()
{
	int t[] = {1,2,3,4,5};
	std::cout << reduce(t,5,0,&add) << std::endl;
	std::cout << reduce(t,5,0,&mul) << std::endl;
}
		\end{lstlisting}
		
		Itt \texttt{reduce} egy olyan paramétert is vár, mely igazából egy függvény, mely \texttt{int}-et ad vissza, és két \texttt{int}-et vár paraméterül.
		\medskip
		
		A kódban feltűnhet, hogy a tömb mellé paraméterben elkértük annak méretét is. Ez azért van, mert a \texttt{t} tömb egy \texttt{int}-re mutató mutatóvá fog konvertálni, ami az első elemre mutat. Ennek hatására értelemszerűen elvesztjük azt az információt, hogy mekkora volt a tömb. Így át kell adni ezt az információt is. Mellékesen, függvényeket kezelni cska pointerekkel lehet, és mivel a fordító tudja hogy függvényeket akarunk átadni, így a \texttt{\&} jel elhagyható függvényhíváskor, és az\texttt{op} elől is elhagyható a * a paramétereknél.
		\begin{lstlisting}
int reduce(int *start, int size, int initial, int op(int, int)))
{
	//...
}

int main()
{
	int t[] = {1,2,3,4,5};
	std::cout << reduce(t,5,0,add) << std::endl;
	std::cout << reduce(t,5,0,mul) << std::endl;
}
		\end{lstlisting}
	\subsection{Tömbökről bővebben}
		\begin{lstlisting}
#include <iostream>

int main()
{
	int t[] = {5,4,3,2,1};
	int *p = t;
	std::cout << *p << std::endl;
	std::cout << sizeof(int) << std::endl;
	std::cout << sizeof(p) << std::endl;
	std::cout << sizeof(t) << std::endl;
}
		\end{lstlisting}
		Könnyű azt hinni (hibásan) hogy a pointerek ugyanazok mint a tömbök. Ez  program jól mutatja, hogy ez nem igaz, mert a tömb tárolja annak méretét is. Számos más különbség is van, viszont egy tömb könnyen konvertálódik pointerré.
		
		Egy tömb adott elemét sokféleképpen le tudjuk kérdezni:
		
		{\centering \texttt{*(p + 3) == *(3 + p) == p[3] == 3[p]} \par}
		\
		\begin{lstlisting}
#include <iostream>

int main()
{
	int t[][3] = {{1,2,3},{4,5,6}};
	return 0;
}
		\end{lstlisting}
		Ez egy alternatív módja egy tömb inicializálásának. Itt több dolog megfigyelendő: Az első \texttt{[]} operátorban nincs méret, mert a fordító az inicializáció alapján meg tudja állapítani, hogy a mátrix azon dimenziója mekkora, de annyira már nem okos, hogy a másodikat is abszolválja.
		
		\medskip
		Fontos megjegyezni, hogy a mátrix egy adott elemére még többféleképpen tudunk hivatkozni:
		
		{\centering \texttt{t[1][] == *(*(t+1)+0) == *(1[t]+0) == 0[1[t]] == 0[*(t+1)] == *(t+1)[0] == 1[t][0] } \par}
	\begin{note}
		Ahhoz, hogy egy olyan függvényt írjunk ami minden méretű tömböt elfogad paraméterül, a legegyszerűbb megoldás, ha hagyjuk, hogy a tömb átkonvertáljon egy olyan pointerré, ami az első elemre mutat, és átadjuk külön paraméterben a tömb méretét. Bár van megoldás arra is, hogy egy darab "rugalmas" függvényt írjunk, és az egész tömbhöz csak 1 paramétert vegyünk át, annak is komoly hátulütői lehetnek. Majd template-ekkel a 7-8. gyakorlaton lesz részletesen szó, de kb így nézne ki:
		\begin{lstlisting}
template <class T, int ArraySize>
void ( T (&param)[ArraySize] )
{
	//...
}
		\end{lstlisting}
		\smallskip
		Ez később jobban ki lesz fejtve, de itt egy template paraméter dedukció fog létrejönni, és a fordító kitalálja \texttt{param} méretét. Csak nyilván, mindig amikor egy más méretű tömböt hozunk létre, a fordító példányosítja ezt a függvényt, ami csúnyán meg tudja dobni a bináris kódot (\textit{binary code}).
		
		\smallskip
		A tömbök átvétele paraméterként azért ilyen körülményes, mert egy tömbnek a méretét fordítási időben ismernünk kell. Ha változó méretű tömböt várnánk paraméterül, az szembemenne ezzel a követelménnyel.  
	\end{note}
	\begin{note}
		Progalapon úgy tanultunk tömböket, hogy változó méretet adtunk meg nekik. Ezt a \texttt{gcc} compiler elfogadja, lefordítja, és jó kódot is csinál belőle, de nem garantált, hogy ezt minden fordító megteszi, ugyanis a c++ szabvány azt mondja ki, hogy a tömb méretének fordítási időben ismertnek kell lennie. Ez jól demonstrálja, hogy a compilerek nem feltételnül követik szorosan a szabványt.
	\end{note}
\end{document}
